\documentclass[11pt]{scrartcl}
\usepackage[sexy,noasy]{evan}
\author{Konrad Kaczmarczyk}
\usepackage{amsmath,systeme}
\usepackage{listings}
\usepackage[T1]{fontenc}
\begin{document}
    \title{Rachunek Prawdopodobieństwa I*}
    \subtitle{Rozwiązanie zadania domowego nr.1}
    \maketitle
    \begin{zadanie*}
        Niech $A_k$ będzie zbiorem liczb naturalnych podzielnych przez $k$. 
        Wykazać że na $\NN$ nie istnieje takie \textbf{P}, że 
        $\textbf{P} \left ( A_k \right ) = \frac{1}{k}$ dla $k \geq 1$.  
    \end{zadanie*}

    Przez sprzeczność oznaczmy że tak nie jest, w takim razie oznaczmy, 
    że $p_n$ to n-ta liczba pierwsza, i mamy :
    \[
    \sum_{n=1}^{\infty } \mathbb{P}(A_{p_n}) = \sum_{n=1}^{\infty }  \frac{1}{p_n} = \infty   
    \]

    jest to fakt z ćwiczeń Analizy I.1, a moim ulubionym dowodem jest ten 
    pana James'a  A. Clarkson'a z 1965 :

    https://fermatslibrary.com/s/on-the-series-of-prime-reciprocals

    \vspace{10px}
    
    skoro tak jest to pierwsze z założeń drugiego lematu Borel-Cantelli jest spełnione, 
    a drugie o niezależności $A_{p_n}$ w naturalny sposób też jest.

    Wynika więc z lematu że:
    \[
        \mathbb{P} \left ( \limsup_{n \to \infty } A_{p_n} \right ) = 1
    \]

    lecz łatwo zauważyć że żadna liczba $x$ nie może należeć do 
    $\limsup_{n \to \infty } A_{p_n}$ bo istnieje takie $N$ że 
    $\forall_{n \geq N} \; x < p_N < p_n \lthen x \not \in A_{p_n}$, 
    więc mamy sprzeczność w postaci:
    \[
        0 = \mathbb{P} \left ( \varnothing \right ) = 
        \mathbb{P} \left ( \limsup_{n \to \infty } A_n \right ) 
        = 1
    \]

    \newpage

    \begin{zadanie*}
        Uzasadnij, że gdy \( k \to \infty \) oraz \( n - k \to \infty \), to
        \[
            p_{k,n} (\pi k^{\frac{1}{2}} (n - k)^{\frac{1}{2}}) \to 1.
        \]

        Przypomnijmy, że \( p_{k,n} = u_k u_{n-k} \) oraz \( u_n = \binom{2n}{n} 2^{-2n} \). 
        Wywnioskuj stąd, że dla \( \alpha \in (1/2, 1) \)

        \[
            \sum_{n/2 < k < \alpha n} p_{k,n} \sim 
            \frac{1}{\pi n} \sum_{n/2 < k < \alpha n} 
                \left( \frac{k}{n} (1 - \frac{k}{n}) \right)^{-\frac{1}{2}}.
        \]
        Nadto

        \[
            \frac{1}{\pi n} \sum_{n/2 < k < \alpha n} 
                \left( \frac{k}{n} (1 - \frac{k}{n}) \right)^{-\frac{1}{2}} \sim 
            \pi^{-1} \int_{1/2}^{\alpha} \frac{dx}{(x(1-x))^{\frac{1}{2}}} = 
            2 \pi^{-1} \arcsin \alpha^{\frac{1}{2}} - \frac{1}{2}.
        \]
    \end{zadanie*}

    Zauważmy że z wzoru Stirlinga wynika że:
    \[
        u_n = 2^{-2n} \binom{2n}{n} = 2^{-2n} \frac{(2n)!}{(n!)^2} \sim 
        2^{-2n} \frac{ \sqrt{4 \pi n} \left ( \frac{2k}{e} \right )^{2n}}
            {2 \pi k \left ( \frac{k}{e} \right )^{2k}}
        = \frac{1}{\sqrt{\pi k} }
    \]
    przy $n \to \infty$, więc otrzymujemy oszacowanie na $p_{k,n}$:
    \[
            p_{k,n} = 
            u_k u_{n-k} \sim
            \frac{1}{\sqrt{\pi k} } \, \cdot \, \frac{1}{\sqrt{\pi (n-k)} } =
            \pi^{-1} \, \left( n (n - k) \right)^{-\frac{1}{2}}
    \]

    Z treści zadania mamy znaleźć asymptotykę pewnej sumy, 
    zatem korzystając z wcześniejszego rachunku:
    \begin{gather*}
            \sum_{n/2 < k < \alpha n} p_{k,n} = 
            \sum_{n/2 < k < \alpha n} u_k u_{n-k} \sim
            \sum_{n/2 < k < \alpha n} \frac{1}{\sqrt{\pi k} } \, 
                \cdot \, \frac{1}{\sqrt{\pi (n-k)} } = \\
            \frac{1}{\pi n} \sum_{n/2 < k < \alpha n} 
                \left( \frac{k}{n} (1 - \frac{k}{n}) \right)^{-\frac{1}{2}}
    \end{gather*}
    Warto zauważyć że ostatnie jest sumą Riemana całki z $\frac{1}{\sqrt{x(1 - x)}}$, czyli:

    \begin{gather*}
         \frac{1}{\pi n} \sum_{n/2 < k < \alpha n} 
             \left( \frac{k}{n} (1 - \frac{k}{n}) \right)^{-\frac{1}{2}} \sim
             \pi^{-1} \int_{\frac{1}{2}}^{\alpha} \frac{dx}{\sqrt{x(1-x)} }
    \end{gather*}
    całkę tę można rozwiązać poprzez podstawienie $x = \sin^2 u$:

    \[
        = \pi^{-1} \int_{\frac{\pi }{4}}^{\arcsin \sqrt{\alpha }} 2 du =
        \frac{2}{\pi } \left ( \arcsin \sqrt{\alpha }  - \frac{\pi }{4} \right ) = 
        \frac{2}{\pi } \arcsin \sqrt{\alpha } - \frac{1}{2}
    \]
    co mieliśmy udowodnić.
\end{document}
