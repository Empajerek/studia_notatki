\documentclass[11pt]{scrartcl}
\usepackage[sexy]{evan}
\author{Konrad Kaczmarczyk}
\usepackage{amsmath,systeme}
\usepackage{listings}
\usepackage[T1]{fontenc}
\begin{document}
  \title{JAiO lato 2024}
  \subtitle{notatki z ćwiczeń}
  \maketitle
  \section{27 luty}
  \begin{zadanie}
      Znajdź liczbę wszystkich języków na alfabetem $\{ a, b \}$ o następujących własnosciach:
      \begin{enumerate}
          \item Wszystkie słowa są długosci c o najwyżej $n$
          \item Wszystkie słowasą długosci $n \geq 2 $ i każde słowo zawiera in fiks $aa$
          \item Słowa nie zawierają symbolu $b$.
          \item Język jest skończony.
      \end{enumerate}
  \end{zadanie}
  
  \begin{enumerate}
    \item Wystaryczy zauważyć że liczba słów to $\sum_{i = 0}^{n} 2^i = 2^{n+1} -1$
      Więc języków jest $\lvert \alpha \rvert = 2^{2^{n+1} - 1}$
    \item Fibonacci
    \item Słów jest przeliczalnie nieskończenie, więc języków jest $2^{\aleph_0}$.
    \item $\aleph_0$ bo możemy języki możemy ustawić względem ich najdłuższego słowa
  \end{enumerate}

  \begin{zadanie}
    Udowodnij że poniższe definicje pewnego języka $L$ nad alfabetem $\{ ( , ) \}$ są równoważne:
    \begin{enumerate}
        \item Słowo puste należy do $L$ oraz jesli $w , v \in L $ to również $wv \in L$ i $(w) \in L$.
        \item $L$ jest zbiorem słów $w$ o następującej własnosci: liczby wystąpień $($ oraz $) $w słowie $w$ sątakie same oraz w dowolnym prefiksie $w$ liczba wystąpień $($ jest większa lub równa liczbie wystąpień $)$.
      
    \end{enumerate}
  \end{zadanie}
      \begin{enumerate}
          \item $L \subset P$
            \begin{gather*}
                \eps \in P \\
                w \in P \\
                (w) \in P \\
                v, w \in L \\
                vw \in B
            \end{gather*}
          \item $P \subset L$
            Indukcja po długosci słowa, i szukamy pierwszego miejsca gdy liczba nawiasów otwierających jest taka sama jak zamykających, i dwa przypadki które idą z założenia.
      \end{enumerate}
  \begin{zadanie}
    Słowo $w \in \sum^*$ nazwiemy \textit{pierwotnym} jesli nie istnieje słowo 
  \end{zadanie}

  \section{5 marca}
      
  \begin{zadanie}
      Reprezentujemy liczby binarne jako słowa nad alfabetem $\left \{ 0, 1 \right \}$, wyraź z pomocą wyrażenia regularnego słowa podzielne przez 3 
  \end{zadanie}
  
  Rozwiązaniem jest $1\left ( 01^*0 \right ) 1 \left ( 1 \left ( 01^*0 \right )^* 1 + 0 \right )^*$, albo prosciej (jednak niepoprawnie) jako $(1(01^*0)^*1+ 0)^*$, gdzie wystarczy pokazać że:
  \begin{walk}
      \item wyrażenie generuje liczby podzielne przez 3
      \item wszystkie liczby podzielne przez 3 mogą być wygenerowane przez wyrażenie
  \end{walk}
  
  \begin{zadanie}
      Roztrzygnij czy języków regularnych jest przeliczalnie czy nieprzeliczalnie wiele.
      \begin{uwaga}
          Nie mamy ustalonego alfabetu, ale przyjmujemy że każdy alfabet będzie skończonym podzbiorem pewnego przeliczalnego zbioru.
      \end{uwaga}
  \end{zadanie}

  Każdy alfabet jest sk. podzbiorem $\left \{ a_{1}, a_{2}, a_{3}, \cdots  \right \}$
  Rozważmy następujące skończone zbiory:
  \begin{enumerate}
      \item wszystkie wyr. reg. długosci $\leq 1$ , i wykorzystujące litery $\left \{ a_{1} \right \}$
      \item ------------||-------------- $\leq 2$ , -----------||--------- $\left \{ a_{1}, a_{2} \right \}$
      \item \dots 
  \end{enumerate}
  
  I zauważmy że każdy krok jest skończony, i każde słowo znajdzie się jakiejs kategorii.

  \begin{przykład}
      Jaką głębokosć gwiazdkowa ma $\left ( a^* b^* \right )^*$ ?
  \end{przykład}
  
  Zauważmy że: $\left ( a^* b^* \right )^* = (a+b)^*$, więc 1.

  \begin{zadanie}
      Udowodnij, że dla każdego $k \in \NN$ istnieje język regularny o głębokości gwiazdkowej $k$. Czy można ograniczyć się do alfabetu z dwoma symbolami? A z jedym symbolem? Co się dzieje jeśli dopuścimy do wyrażeń regularnych dopełnienie?
  \end{zadanie}

  Weźmy alfabet $\left \{ a_{1}, a_{2}, a_{3}, \cdots , a_{k} \right \}$, i rozważmy:
  \[
    \left ( a_k \left ( a_{k-1} \left ( \dots a_{1}^* \right )^* \right )^* \right )^*
  \]
  i spróbować wykazać że ma on porządaną głębokosć gwiazdkową.

\end{document}
