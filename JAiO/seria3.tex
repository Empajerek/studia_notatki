\documentclass[11pt]{scrartcl}
\usepackage[sexy]{evan}
\author{Konrad Kaczmarczyk}
\usepackage{amsmath,systeme}
\usepackage{listings}
\usepackage[T1]{fontenc}
\begin{document}
  \title{JAiO lato 2024}
  \subtitle{rozwiązanie zadań z serii III}
  \maketitle
    \section{Zadanie}
        \begin{zadanie}
            Ustalmy alfabet wejściowy $A = \{0, 1\}$ i alfabet taśmowy $T = \{0, 1, \mathbb{B} \}$. Niech U będzie maszyną uniwersalną. Dla słowa $w \in A^*$ zdefiniujmy maszynę Turinga $M_w$ następująco: jeśli w jest kodem pewnej maszyny $M$ to $M_w := M$ , a w przeciwnym przypadku $M_w := U$ . Dla każdego z poniższych języków ustal czy jest rozstrzygalny, i czy jest częściowo rozstrzygalny. 
            \begin{walk}
                \item $L_1 = \{ w \in A^* \mid M_w \; \text{jest maszyną o liczbie stanów, która jest potęgą dwójki} \}$
                \item $L_2 = \{ w \in A^* \mid \forall v \in L(M_w) \exists n \in \mathbb{N} v = 1^{2^n}  \}$
                \item $L_3 = \{ w \in A^* \mid \forall v \in L(M_w) v = 1^{\abs{w} } \vee \exists n \in \mathbb{N} v = 1^{2^n}  \}$  
            \end{walk}
            
        \end{zadanie}
        
        \section{Rozwiązanie}

          \subsection{a)}
            
            Jest to język rostrzygalny, udowodnimy to przez skonstruowanie automatu rozpoznającego ten język.

            Możemy skonstruować maszynę $P$, rozpocznającą poprawnosć kodu $w$, następnie jęsli kod jest poprawny przekazujemy go do maszyny zliczającej stany w kodzie $w$, i przekazujemy ich ilosć do maszyny sprawdzającej czy jest to potęga dwójki (poprzez dzielenie jej przez 2, aż dojdziemy do 1, chyba że jakas liczba nie jest podzielna).

          \subsection{b)}

            Na początku pokażemy, że ten język nie jest rozstrzygalny.
            Przez sprzeczosć, załużmy że jest roztrzygalny, czyli jest rozpoznawany przez np. całkowitą maszyne $P$, i wynika z tego równierz rostrzygalnosć problemu stopu.
            Weźmy dowolne słowo $w$ i maszyne $Q$. Niech maszyna $R$, niezależnie od swojego wejscia symuluje bieg słowa $w$ w maszynie $Q$, i jesli jest rozpoznawany daje na wyjsciu $1$, w pozostałym przypadku niech się zawiesza (podobnie jak $Q$). Ale teraz uruchamiamy kod maszyny $R$ na maszynie $P$, i jesli $R$ dla każdego słowa daje $1$, to kod maszyny nie jest rozpoznawalny przez $P$. Łącząc te rzeczy mamy że $P$ rozpoznaje tylko kody maszyn symulujących zawieszające się układy maszyn i słów, co rozstrzyga problem stopu, więc mamy sprzecznosć.

            Teraz pokażemy, że ten język nie jest częsciowo rozstrzygalny.
            Wystarczy pokazać, że dopełnienie tego języka $A^* - L_2$ jest częsciowo rostrzygalne (teza wtedy wynika z poprzedniego wniosku), czyli kodów maszyn które zawierają conajmniej jedno słowo $v \not = 1^{2^n}$. Mianowicie aby sprawdzić czy $M_w \in A^* - L_2$, możemy symulować jej działanie przez $k$ kroków na kolejno wszystkich słowach o długosci $\leq k$, stopniowo zwiększając $k$ gdy skończą nam sie słowa, gdy znajdziemy takie słowo akceptujące $\not = 1^{2^n}$ to kończymy. Widać że jest to problem częsciowo rostrzygalny, co kończy dowód.

            \subsection{c)}
            
            Ten podpunkt nie różni się w zasadnie niczym od poprzedniego, wykazanie że nie jest rozstrzygalny jest identyczne jak poprzednio, a w dowodzie braku cz. rostrzygalnosci do każdego $v \not = 1^{2^n}$, dokładamy warunek że $v \not = 1^{\abs{w} }$ 
\end{document}
