\documentclass[11pt]{scrartcl}
\usepackage[sexy]{evan}
\author{Konrad Kaczmarczyk}
\usepackage{amsmath,systeme}
\usepackage{listings}
\usepackage[T1]{fontenc}
\begin{document}
  \title{JAiO lato 2024}
  \subtitle{Rozwiązania zadań z serii II}
  \maketitle
   \section{Zadanie}
       \begin{zadanie}
           Dla języka $L \in \Sigma^*$ definiujemy:
           \[
             \sqrt{L} := \{ w \in \Sigma \; ww \in L \}
           \]
           \begin{walk}
               \item Czy język $L$ jest bezkontekstowy, to język $\sqrt{L} $ jest bezkontekstowy?
                \item Czy jesli język $\sqrt{L} $ jest bezkontekstowy, to język $L$ jest bezkontekstowy?
                \item Czy odpowiedzi zmieniają się, gdy ograniczymy się do alfabetów dwu- lub jednoliterowych?
           \end{walk}
           
       \end{zadanie}
       
       \begin{walk}
           \item Nie, weźmy za przykład $L = a^n b^n a^* b^m a^m$ (który jest bezkontekstowy jako konkatenacja języków $a^n b^n$, $a^*$, $b^m a^m$ które są oczywiscie bezkontekstowe), dla którego język $\sqrt{L} $ to 
             \[
                 \sqrt{L} = a^n b^n a^n 
             \]
             który już bezkontekstowy nie jest (z lematu o pompowaniu dla języków bezkontekstowych).
           \item Nie, weźmy za przykład $L = \{ a^p \; | \; p \in \mathbb{P} \}$ (który oczywiscie nie jest bezkontekstowy), ale za to
             \[
               \sqrt{L} = \{ \varepsilon , a \} 
             \]
             jest bezkontekstowy, co zaprzecza.

          \item Widzimy że dla ograniczając się do automatów jedno- i dwuliterowych teza w podpunkcie (b) pozostaje taka sama, oraz że dla dwuliterowych w podpunkcie (a) też. Pozostaje zatem pokazać że dla jednoliterowych teza pierwszego się zmieni.

            Zatem, korzystając z tw. Parikh'a mamy i że nasz alfabet jest jednoliterowy mamy że dowolny język bezkontekstowy $L$ jest regularny, ale wtedy język $\sqrt{L} $ też jest regularny (fakt ten pojawił się na ćwiczeniach \url{https://www.mimuw.edu.pl/~lk406698/teaching/JAiO2024/tutorial3.pdf} ), a więc bezkontekstowy, co kończy dowód.
       \end{walk}
       
   
\end{document}
