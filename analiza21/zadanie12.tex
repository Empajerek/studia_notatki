\documentclass[11pt]{scrartcl}
\usepackage[sexy,noasy]{evan}
\author{Konrad Kaczmarczyk}
\usepackage{amsmath,systeme}
\usepackage{listings}
\usepackage[T1]{fontenc}
\begin{document}
  \title{Analiza II.1*}
  \subtitle{Rozwiązanie zadania domowego nr. 12}
  \maketitle
    \begin{zadanie*}
        Dane jest 10 zbiorów. Ile nowych zbiorów można z nich skonstrować za pomocą (wielokrotnego) stosowania operacji: przecięcie, sumy, różnicy, i sym. różnicy?
    \end{zadanie*}
    
    Oczywistym jest że każdy element albo należy do zbioru albo nie, zatem dla każdego elementu w $A_1 \cup \dots  \cup A_{10}$ istnieje ciąg liter $T, N$ mówiący czy zbiór należy do konkretnego zbioru np: $TTTTTTTTTT$ odpowiada elementom zawierającym się w przecięciu wszystkich zbiorów, a $TNNNNNNNNN$ odpowiada elementom zawierającym się tylko w $A_1$ (zanaczmy że ciąg $NNNNNNNNNN$ odpowiada zbiorowi pustemu). 


    Zauważmy że relacja $\sim$ definiowana przez ten sam ciąg liter jest relacją równoważnosci względem operacji przecięcia, sumy itd. (co można pokazać dla każdej operacji z osobna), zatem możemy klasę równoważności traktować jako jeden element. 
    
    Czyli możemy sytułacje uprościć do przypdku gdzie elementów jest $2^{10} - 1$ (ile możliwych ciągów liter z zbiorem pustym), oraz postawić że wynik to $2^{2^{10}-1}$ gdyż łatwo skonstuować dla dowolnego elementu singletony (wystarczy odpowiednia suma i przecięcie), a z nich skonstruować dowolny zbiór elementów.
\end{document}
