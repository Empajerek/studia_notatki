\documentclass[11pt]{scrartcl}
\usepackage[sexy,noasy]{evan}
\author{Konrad Kaczmarczyk}
\usepackage{amsmath,systeme}
\usepackage{listings}
\usepackage[T1]{fontenc}
\begin{document}
  \title{Analiza Matematyczna II.1*}
  \subtitle{Rozwiązanie zadania domowego nr.14}
  \maketitle
    \begin{zadanie*}
        Pokazać nierówność
        \[
          \text{exp} \left ( \frac{1}{b - a} \int_{[a,b]} f dl_1  \right ) \leq \frac{1}{b - a} \int_{[a,b]} \text{exp} (f) dl_1
        \]

        gdzie $f$ jest mierzalna i całkowalna na $[a,b]$.
        
    \end{zadanie*}

    Niech:
    \[
      x_0 = \frac{1}{b - a} \int_{[a,b]} \text{exp} (f) dl_1
    \]

    z pierwszego semestru analizy znamy fakt że styczna do wykresu funkcji wypukłej jest "pod" nim, czyli:

    \[
      \exists_{c, d} \; e^x > cx + d \; \wedge \; e^x_0 = cx_0 + d
    \]

    czyli mamy nierówności:

    \begin{gather*}
      \int_{[a, b]} \text{exp} (f) dl_1 \geq \int_{[a,b]} cx + d dl_1 \\
      = c \int_{[a,b]} f dl_1 + d (b - a) = (b - a)(cx_0 + d) = (b - a) \; e^x_0 \\
      = (b - a) \text{exp} ( \frac{1}{b - a} \int_{[a,b]} f dl_1 )
    \end{gather*}

    \fbox{\rule{1.3in}{0pt} \tiny miejsce na tensora \rule[-0.5ex]{0pt}{2in}}
    
\end{document}
