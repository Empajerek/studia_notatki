\documentclass[11pt]{scrartcl}
\usepackage[sexy,noasy]{evan}
\author{Konrad Kaczmarczyk}
\usepackage{amsmath,systeme}
\usepackage{listings}
\usepackage[T1]{fontenc}
\begin{document}
  \title{Anaiza II.1*}
  \subtitle{Rozwiązanie zadania domowego nr. 10}
  \maketitle
    \begin{zadanie*}
        Elipsą na płaszczyźnie $\RR^2$ nazywamy:
        \[
            E = \left \{ \frac{x^2}{a^2} + \frac{y^2}{b^2} = 1 \right \}
        \]
        Z punktu $(x,y) \in E$ wypuszczany normalną $N$. Jaka jest największa odległość $N$ od $(0, 0)$?
    \end{zadanie*}
    
    Niech $f : \RR^2 \to \RR$, będzie funkcją dla której elipsa $E$ jest poziomicą. Z poprzednich zadań domowych wiemy że funkcje $f$ możemy zapisać jako $f(x, y) = g(b^2 x^2 + a^2 x^2)$, w szczególności $f = b^2 x^2 + a^2 y^2$. Oznaczmy punkt wypuszczenia normalnej $(x_0, y_0)$, i mamy że normalna $N \bot E \bot (\nabla f)(x_0, y_0)$ czyli (korzystając z faktu że znajdujemy się w $\RR^2$) $N \parallel (\nabla f)(x_0, y_0)$. Łącząc jest $N = \alpha \left ( \nabla f \right )(x_0, y_0) + (x_0, y_0)$, przy $\alpha \in \RR$.
    \begin{gather*}
      (\nabla f)(x_0) \alpha + (x_0, y_0)  = 
      \begin{bmatrix}
          2 b^2 x_0 \\ 2 a^2 y_0
      \end{bmatrix} \alpha + (x_0, y_0)
      = \left \{(x,y) \in \RR^2 : 2 (a^2 - b^2) x_0 y_0 = 2 a^2 y_0 x - 2 b^2 x_0 y \right \}
    \end{gather*}
    
    upraszczając, mamy:
    \[
      N = \left \{ (x,y) \in \RR^2 : a^2 - b^2 = a^2 \frac{x}{x_0} + b^2 \frac{y}{y_0}   \right \}
    \]
    czyli z wzoru na odległość punktu od prostej mamy że odległość to
    \[
        \frac{\abs{a^2 - b^2} }{\sqrt{\left ( \frac{a^2}{x_0} \right )^2 + \left ( \frac{b^2}{y_0} \right )^2 }  }
    \]

    pozostało nam zadanie optymalizacyjne, skoro $\abs{a^2 - b^2} $ jest stałą, to odległość jest największa gdy wyrażenie w mianowniku jest najmniejsze (pierwiastek jest funcją rosnącą więc też można go pominąć), oznaczając $\frac{x^2}{a^2} \to x$, i $\frac{y^2}{b^2} \to y$, mamy:
    \[
        V = \frac{a^2}{x} + \frac{b^2}{y} \qquad G = x + y - 1 = 0
    \]
    poszukujemy więc mininum funkcji $V$, przy warunku $G = 0$. Korzysytając z mnożników lagrange'a mamy:
    \begin{gather*}
        - \frac{a^2}{x^2} + \lambda = 0 \\
        - \frac{b^2}{y^2} + \lambda = 0 \\
        x + y = 1
    \end{gather*}
    po podstawieniu mamy że $\min V = (a+b)^2$ ( wiemy że jest to minimum chociażby ze struktury $V$), i powracając do odległości punktu mamy że:
    \[
        \min d(N, (0, 0)) = \frac{\abs{a^2 - b^2}  }{\abs{a + b} } = \abs{a - b} 
    \]
    
\end{document}
