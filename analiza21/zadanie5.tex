\documentclass[11pt]{scrartcl}
\usepackage[sexy]{evan}
\author{Konrad Kaczmarczyk}
\usepackage{amsmath,systeme}
\usepackage{listings}
\usepackage[T1]{fontenc}
\begin{document}
  \title{Analiza II.1*}
  \subtitle{Rozwiązanie zadania domowego nr.5}
  \maketitle
    \begin{zadanie*}
        Niech $U = \left \{ x > 0, y > 0 \right \} \subset \mathbb{R}^2$, z przekształceniem \\ $f : U \to \mathbb{R} $ różniczkowalnym i spełniającym:
        \[
            x \frac{\partial}{\partial x} f(x,y) = 7 y \frac{\partial}{\partial y} f(x,y) 
        \]
        Pokazać, że istnieje $g: \mathbb{R}_+ \to \mathbb{R} $ różniczkowalne, i takie że 
        \[
            f(x,y) = g(x^7 y)
        \]
        
    \end{zadanie*}
    
    Rozwiązanie rozpoczniemy od podstawienia nowych zmiennych $x = e^u$, $y = e^w$, \\ i $\partial x = x \; \partial u$ i $\partial y = y \; \partial w$:
    \begin{gather*}
          x \frac{\partial}{\partial x} f(x,y) = 7 y \frac{\partial}{\partial y} f(x,y) \\
          \frac{\partial}{\partial u} f(e^u, e^w) = 7 \frac{\partial}{\partial w} f(e^u, e^w)
    \end{gather*}
    nazwijmy funkcją $h$ funkcje $h := f(e^u, e^w)$, i mamy że 
    \[
        0 = \frac{\partial}{\partial u} h(u,w) - 7 \frac{\partial}{\partial w} h(u,w) = \left ( \nabla h \right )
        \begin{bmatrix}
          1 \; -7
        \end{bmatrix}
    \]
    
    Z czego wynika że jesli:
    \[
        h(u_0, w_0) = 0 \qquad \Rightarrow \qquad \forall_\alpha \;  h((u_0, w_0) + \alpha (1, -7)) = c
    \]
    czyli poziomice to proste $7x + y = k$, więc funkcje $h$ możęmy zapisać w postaci $h(u,w) = k(7u + w)$, i wracając do funkcji $f$ mamy:
    \[
        f(x,y) = h \left (\text{ln} \left ( x \right ), \text{ln} \left ( y \right )  \right ) =
        k \left ( 7\text{ln} \left ( x \right ) + \text{ln} \left ( y \right )  \right ) = 
        k \left ( \text{ln} \left ( x^7 y \right )  \right ) = g(x^7 y)
    \]
\end{document}
