\documentclass[11pt]{scrartcl}
\usepackage[sexy,noasy]{evan}
\author{Konrad Kaczmarczyk}
\usepackage{amsmath,systeme}
\usepackage{listings}
\usepackage[T1]{fontenc}
\begin{document}
  \title{Analiza 2.1*}
  \subtitle{Rozwiązanie zadania domowego nr.8}
  \maketitle
    \begin{zadanie*}
      \[
        SO(3) = \{ A \in M_{3 \times 3}(\mathbb{R}) : A A^\top = I_d, \det A = 1 \}
      \]

      Pokaż, że \( SO(3) \) jest podrozmaitością gładką wymiaru \( 3 \) w \( \mathbb{R}^9 \).
    \end{zadanie*}

    Niech $F(A) = A A^T - I \in S_{3 \times 3}$. $F$ spełnia warunek twierdznia o funkcji odwrotnej czyli:
    \begin{gather*}
        F (A + H) - F(A) = (A + H) (A + H)^T - I - (AA^T - I) \\ = AA^T - AA^T + HA^T + AH^T \\ = HA^T + AH^T + HH^T \\ = F'(A) H + HH^T
    \end{gather*}
    Z warunku twierdzenia mamy że
    \[
      \forall_{S \in S_{3 \times 3}} \exists_{H \in M_{3 \times 3}} HA^T + AH^T = S
    \]
    czyli $H = \half S A$. A zatem rząd macierzy przekształcenia $F$ to $6$, (warto zauważyć że warunek $\opname{det} A = 1$ wybiera tylko spójną składową wśród macierzy ortogonalnych), czyli z twierdzenia o rzędzie rozmaitość gładka $SO(3)$ jest wymiaru $3$. 

    
\end{document}
