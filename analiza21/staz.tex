\documentclass[11pt]{scrartcl}
\usepackage[sexy,noasy]{evan}
\author{szymon ulica™}
\usepackage{amsmath,systeme}
\usepackage{listings}
\usepackage[T1]{fontenc}
\begin{document}
    \begin{zadanie*}
        Punkty na płaszczyźnie są pokolorowane jednym z trzech kolorów: czerwonym, zielonym lub niebieskim. Udowodnij, że dla danej odległości d zawsze istnieją dwa punkty tego samego koloru w odległości d od siebie.
    \end{zadanie*}

    \vfill
    \begin{zadanie*}
    Niech $\mathfrak{M}$ będzie $\sigma$-ciałem nad zbiorem $\Omega$. Załóźmy że $\mathfrak{M}$ jest przeliczalny. Udowodnij że $\abs{\mathfrak{M}} $ jest skończone i potęgą liczby 2.
    \end{zadanie*}
    \vspace{0.4\textheight}
    
\end{document}
