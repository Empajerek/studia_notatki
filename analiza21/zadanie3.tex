\documentclass[11pt]{scrartcl}
\usepackage[sexy]{evan}
\author{Konrad Kaczmarczyk}
\usepackage{amsmath,systeme}
\usepackage{listings}
\usepackage[T1]{fontenc}
\begin{document}
  \title{Analiza 2.1*}
  \subtitle{Rozwiązanie zadania domowego nr.3}
  \maketitle
    \begin{zadanie*}
        Rozważmy funkcje 
        \[
          f(x,y) = \frac{\left ( e^{x+y} - 1 \right ) \text{sin} \left ( x - y \right ) }{x^2 - y^2}
        \]
        dla $\abs{x} \not = \abs{y}  $
        \begin{walk}
            \item Pokazać, że $f$ można przedłużyć do funkcji ciągłej na $\mathbb{R}^2$.
            \item Czy to przedłużenie jest różniczkowalne?
        \end{walk}
        
    \end{zadanie*}
    
    \begin{walk}
        \item Przeprowadźmy zamianę zmiennych na:
          \[
            \begin{cases} s = x - y \\ t = x + y \end{cases}
          \]
          wtedy nasza funkcja to 
          \[
              f(s,t) = \frac{e^t - 1}{t} \cdot \frac{\text{sin} \left ( s \right ) }{s}
          \]
          widać że problematycznymi punktami naszej funkcji są $s = 0$ i $t = 0$, lecz znane są granice:
          \[
              \lim_{t \to 0} \frac{e^t - 1}{t} = 1 \qquad \lim_{s \to 0} \frac{\text{sin} \left ( s \right ) }{s} = 1    
          \]
          więc wystarzczy dodefiniować:
          \[
            f(s,t) = \begin{cases} \frac{e^t - 1}{t} \cdot \frac{\text{sin} \left ( s \right ) }{s} \qquad & x \not = 0 \wedge y \not = 0 \\  
              \frac{e^t - 1}{t} = f_1(t) \qquad & s = 0 \wedge t \not = 0 \\
              \frac{\text{sin} \left ( s \right ) }{s} = f_2 (s)  \qquad & t = 0 \wedge s \not = 0 \\
            1 \qquad & s = 0 \wedge t = 0 \end{cases}
          \]
          i nasza funkcja jest ciągła, ponieważ jest iloczynem funkcji ciągłych.

        \item Używając poprzednich oznaczeń, obliczmy dwie pochodne:
          \[
            f_1'(x) = \left ( \frac{e^x - 1}{x} \right )' = \frac{e^x (x - 1) + 1}{x^2} \qquad 
            f_2'(x) = \left ( \frac{\text{sin} \left ( x \right ) }{x} \right )' = \frac{x \cdot  \text{cos} \left ( x \right ) - \text{sin} \left ( x \right ) }{x^2}
          \]

          których granice w zerze to odpowiednio (używając wzorów Taylora):
          \[
              \lim_{x \to 0} \frac{e^x \left ( x - 1 \right ) + 1}{x^2} = \half \qquad 
              \lim_{x \to 0} \frac{x \cdot \text{cos} \left ( x \right ) - \text{sin} \left ( x \right ) }{x^2} = 0  
          \]
          obliczmy teraz $f_1'(0)$ i $f'_2(0)$:
          \begin{gather*}
              f'_1(0) = \lim_{x \to 0} \frac{f_1(x) - 1}{x} = \lim_{x \to 0} \frac{e^x - x - 1}{x^2} = \half \\
              f'_2(0) = \lim_{x \to 0} \frac{f_2(x) - 1}{x} = \lim_{x \to 0} \frac{\text{sin} \left ( x \right ) - x}{x^2} = 0    
          \end{gather*}
          korzystając ze wzoru Taylora, a zatem pochodne zdefiniowanych przez nas funkcji są ciągłe. 
          
          Teraz wracając do naszej funkcji wyjsciowej $f$ mamy że jej pochodne cząstkowe:
          \[
              \frac{\partial f}{\partial s} = f_1 (t) \frac{df_2}{ds} = f_1(t) \cdot f'_2(s) \qquad 
              \frac{\partial f}{\partial t} = f_2 (s) \frac{df_1}{dt} = f_2(s) \cdot f'_1(t)
          \]
          są iloczynami funkcji ciągłych więc są ciągłe, a zatem cała funkcja jest różniczkowalna w każdym punkcie.

    \end{walk}
    
\end{document}
