\documentclass[12pt]{scrartcl}
\usepackage[sexy]{evan}
\author{Konrad Kaczmarczyk}
\usepackage{amsmath,systeme}
\usepackage{listings}
\usepackage[T1]{fontenc}
\begin{document}
  \title{Analiza I.2*}
  \subtitle{Rozwiązanie zadań z serii II}
  \maketitle
    \section{Zadanie}
        
    \begin{zadanie}
        Udowodnij, ze dla $x \in \left ( 0, \frac{\pi }{2} \right )$ prawdziwa jest nierówność
        \[
            \text{cos} \left ( x \right ) < \frac{\text{sin}^2 \left ( x \right ) }{x^2}
        \]
    \end{zadanie}
        Znamy wzór Talora dla funkcji cosinus i $x_{0} = 0$:
        \[
            \text{cos} \left ( x \right ) = 1 - \frac{1}{2} x^2 + \frac{1}{24} x^4 + R_4(0) 
        \]
        Korzystamy z postaci Lagrange'a dla funkcji reszty i mamy
        \[
          R_4 (0) = \frac{\text{cos}^{(5)} \left ( \xi \right ) }{5!} x^4 = - \frac{\text{cos} \left ( x \right ) }{120} x^4 < 0
        \]
        Zatem mamy nierównosć
        \[
            \text{cos} \left ( x \right ) < 1 - \frac{1}{2} x^2 + \frac{1}{24} x^4
        \]
        Możemy zapisać teraz nierównosć:
        \[
            P = \frac{\text{sin}^2 \left ( x \right ) }{x^2} = \frac{\frac{1}{2} - \frac{1}{2} \text{cos} \left ( 2x \right ) }{x^2} > 1 - \frac{1}{3} x^2 > 1 - \frac{1}{2} x^2 + \frac{1}{24} x^4 > \text{cos} \left ( x \right )  
        \]
        po drodze skorzystalismy z nierównosci $\frac{1}{6} x^2 > \frac{1}{24} x^4$, która działa dla warunków zadania.
  \section{Zadanie}
      
    \begin{zadanie}
        Niech $a, b > 0, a \not = b$. Udowodnij nierówność
        \[
            \sqrt{ab} \leq \frac{b - a}{\text{ln} \left ( b \right ) - \text{ln} \left ( a \right ) } \leq \frac{a + b}{2} 
        \]
        
    \end{zadanie}
    
    Będziemy korzystać z lematu o biegaczach. Ustalamy $a,b$, wybierzmy trzy funkcje:
    \begin{gather*}
       g(x) = \frac{b - a}{ \text{ln} \left ( b \right ) - \text{ln} \left ( a \right ) } \cdot \frac{1}{2} x^2 + ax \\     
     f(x) = e^{x + \text{ln} \left ( a \right ) } - a \\
     d(x) = \sqrt{ab} \cdot \left ( \frac{1}{2} x^2 - \frac{\text{ln} \left ( b \right ) - \text{ln} \left ( a \right )  }{2} x \right ) + \sqrt{ab} \cdot x 
    \end{gather*}
    
    Od razu mamy też ich pochodne:
    \begin{gather*}
      g'(x) = \frac{b - a}{\text{ln} \left ( b \right )  - \text{ln} \left (  a \right ) } x + a\\
      f'(x) = e^{x + \text{ln} \left ( a \right ) } \\
      d'(x) =  \sqrt{ab} \cdot \left ( x - \frac{\text{ln} \left ( b \right ) - \text{ln} \left ( a \right )  }{2} \right )  + \sqrt{ab} 
    \end{gather*}
    Łatwo sprawdzić że $f(0) = g(0) > d(0)$
    Zauważmy też geometrycznie że, funkcja $g$ przeciana $f$ w dwóch punktach (dokładnie w $0$, i $\text{ln} \left ( b \right ) - \text{ln} \left ( a \right )  $ ), i faktu że $f$ jest wypukła wynika że na tym przedziale $g' > f'$. Możemy równierz zauważyć że funkcja $d'$ została tak wybrana żeby być styczną do $f'$ w punkcie $\frac{\text{ln} \left ( b \right ) - \text{ln} \left ( a \right )  }{2}$, i znowu z wypukłosci mamy że $f' > d'$, łącząc możemy skorzystać z lematu o biegaczach, i dla punktu $x_{1} = \text{ln} \left ( b \right ) - \text{ln} \left ( a \right )$ mamy:
    \[
        g(x_{1}) \geq f(x_{1}) \geq d(x_1)
    \]
    co po podstawieniu daje nam:
    \[
        \frac{a+b}{2} \cdot \left ( \text{ln} \left ( b \right ) - \text{ln} \left ( a \right )  \right ) \geq b-a \geq \sqrt{ab} \cdot \left ( \text{ln} \left ( b \right ) - \text{ln} \left ( a \right )  \right ) 
    \]
    Wystarczy podzielić i mamy tezę.
\section{Zadanie}
    
    \begin{zadanie}
        Funkcja $f : [0, 1] \to [0, 1]$ jest wypukła i rosnąca oraz $f (0) = 0, f (1) = 1$. Udowodnij, że
        \[
          f(x) \cdot f^{-1} (x) \leq x^2 \qquad x \in [0,1]
        \]
        
    \end{zadanie}
    
    Zapiszmy warunek wypukłosci dla $x_{0} = x, x_{1} = 0$:
    \[
      f( t \cdot f(x) ) \leq t \cdot f(x) \qquad t \in [0,1]
    \]
    Warto jescze zauważyć że funkcja jest "na", co wynika że jest wypukła i osiąga krańce przedziału, zatem możemy zauważyć:
    \[
      f(x) \cdot f^{-1} (x) \leq x^2 \iff f(f(y)) \cdot y \leq \left ( f(y) \right )^2 
    \]
    i możemy przeliczyć:
    \[
        f(f(y)) \cdot y = f(y \cdot \frac{1}{y} \cdot  f(y) ) \cdot y \leq \frac{1}{y} \cdot f(y) \cdot f(y) \cdot y \leq \left ( f(y) \right )^2
    \]
    co kończy dowód.
\newpage
\section{Zadanie}
    
    \begin{zadanie}
        Udowodnij, że dla $x \in \left ( 0, \frac{\pi }{2} \right )$ zachodzi nierówność
        \[
            2 \text{tg} \left ( x \right ) > \text{sinh} \left ( x \right ) 
        \]
        
    \end{zadanie}

    Możemy przekształcić tezę do postaci:
    \[
        x - \text{arc tg} \left ( \frac{\text{sinh} \left ( x\right ) }{2} \right ) > 0 \qquad x \in (0, \infty )
    \]
    Skorzystamy ponownie z lematu o biegaczach dla naszej funkcji. Oznaczmy funkcje po lewej jako $f$, oczywiscie mamy że $f(0) = 0$, i policzmy pochodną:
    \[
        f'(x) = \frac{4 - 2 \text{cosh} \left ( x \right ) + \text{sinh}^2 \left ( x \right )  }{4 + \text{sinh}^2 \left ( x \right ) } = \frac{\left ( \text{cosh} \left ( x \right ) - 1  \right )^2 + 2}{4 + \text{sinh}^2 \left ( x \right ) } > 0
    \]
    Zatem nierównosć zachodzi z lematu o biegaczach.
    
\section{Zadanie}
    
    \begin{zadanie}
        Udowodnij, że dla dowolnych dodatnich liczb rzeczywistych $a_{1}, a_{2}, \dots , a_n$ zachodzą nierówności
        \[
          \sqrt[n]{a_{1} \cdot a_{2} \cdot \dots \cdot a_n} \leq \text{ln} \left ( 1 + \sqrt[n]{\left ( e^{a_1} - 1 \right ) \cdot \dots \cdot \left ( e^{a_n} - 1 \right )}  \right ) \leq \frac{a_{1} + a_{2} + \dots + a_n}{n}
        \]
         
    \end{zadanie}

    Zacznijmy od prawej nierównosci, postawmy $e^{b_i} = e^{a_i} - 1$, mamy 
    \[
      \text{ln} \left ( 1 + e^{\frac{1}{n} \sum_{i = 1}^{n} b_i } \right ) \leq \frac{1}{n} \sum_{i = 1}^{n} \text{ln} \left ( 1 + e^{b_i} \right )  
    \]
   co wynika z twierdzenia Jensena, wystarczy tylko udwodnić że:
   \[
       \left ( \text{ln} \left ( 1 + e^x \right )  \right )^{''} = \frac{e^x}{(e^x + 1)^2} > 0
   \]
   co kończy dowód prawej strony.
   Aby udowodnić lewą nierównosć postawiamy $\text{ln} \left ( 1 + e^{b_i} \right ) = e^{c_i} $ i po przekształceniach mamy:
   \[
     \text{ln} \left ( e^{e^{\frac{1}{n} \sum_{i = 1}^{n} c_i }} - 1 \right ) \leq \frac{1}{n} \sum_{i = 1}^{n} \text{ln} \left ( e^{e^{c_i}} - 1 \right )  
   \]
   które ponownie możemy wywnioskować z twierdzenia Jensena, wystarczy tylko sprawdzić:
   \[
     \left ( \text{ln} \left ( e^{e^x} - 1 \right )  \right )^{''} = \frac{e^{x + e^x} \left ( e^{e^x} + e^x + 1 \right )}{\left ( e^{e^x} + 1 \right )^2} > 0
   \]
   
    
\end{document}
