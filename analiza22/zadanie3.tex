\documentclass[11pt]{scrartcl}
\usepackage[sexy,noasy]{evan}
\author{Konrad Kaczmarczyk}
\usepackage{amsmath,systeme}
\usepackage{listings}
\usepackage[T1]{fontenc}
\begin{document}
\title{Analiza II.1*}
    \subtitle{Rozwiązanie zadań z serii nr. 3}
    \maketitle

    \begin{zadanie*}
        Niech $
        U = \RR^3 \backslash \{ 0 \}, 
        \omega = x \ed y \wedge \ed z + y \ed z \wedge \ed x + z \ed x \wedge \ed y
        $
        oraz $f \in C^\infty (U)$. Wykazać, że $\ed (f(y \; dx - x \; dy))$ jest
        proporcjonalna do formy $\omega$ wtedy i tylko wtedy gdy $f$ jest 
        dodatnio jednorodna stopnia $-2$. Jaki warunek na $f$ jest równoważny 
        równaniu $\ed (f(y \; \ed x - x \ed y)) = 0$?
    \end{zadanie*}

    \begin{zadanie*}
        Niech
        \[
            \omega = \frac{(x-1) \ed y - y \ed x}{(x-1)^2 + y^2} -
            \frac{(x+1) \ed y - y \ed x}{(x+1)^2 + y^2}
        \]
        \begin{walk}
            \item Wykazać, że nie istnieje funkcja 
                $f \in C^\infty \left ( \RR^2 \backslash \{ \pm 1, 0 \} \right )$
                taka, że $\omega = \ed f$.
            \item Czy istnieje funkcja 
                $f \in C^\infty \left ( 
                    \RR^2 \backslash \left \{ [-1, 1] \times 0 \right \} 
                \right )$ taka, że $\omega = \ed f$?
        \end{walk}
    \end{zadanie*}

    \begin{zadanie*}
        Niech $U$ będzie dopełnieniem w $\RR^3$ półprostej 
        $\left \{ x = y = 0, z \leq 0 \right \}$, 
        $W = \left \{ (u,v, w) : w > 0 \right \}$ oraz $\phi : W \to U$ dane jest wzorem
        \[
            \phi (u, v, w) = \left ( uw, vw, \half (-u^2 - v^2 + w^2) \right ).
        \]

        Dana jest 2-forma $\omega \in \Omega^2 (U)$: 
        $\omega = \frac{1}{(x^2 + y^2 + z^2)^{\frac{3}{2}}} 
            ( x \, \ed y \wedge \ed z + y \, \ed z \wedge \ed x + z \, \ed x \wedge \ed y)$
        
        \begin{walk}
            \item Wykazać, że $\phi$ jest dyfeomorfizmem $W$ na $U$.
            \item Obliczyć $\phi* \omega$.
            \item Znaleźć 1-formę $\alpha \in \Omega^1 (W)$ taką, że 
                $\ed \alpha = \phi* \omega$.
            \item Wywnioskować, że istnieje 1-forma $\beta \in \Omega^1 (U)$ 
                taka, że $\ed \beta = \omega$
        \end{walk}
    \end{zadanie*}


\end{document}
