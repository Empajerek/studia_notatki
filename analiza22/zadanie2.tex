\documentclass[11pt]{scrartcl}
\usepackage[sexy,noasy]{evan}
\author{Konrad Kaczmarczyk}
\usepackage{amsmath,systeme}
\usepackage{listings}
\usepackage[T1]{fontenc}
\DeclareMathOperator{\sgn}{sgn}

\begin{document}
    \title{Analiza II.2*}
    \subtitle{Rozwiązanie zadań z serii nr. 2}
    \maketitle

    \begin{zadanie*}
        Niech $f \in L^p \left ( \RR \right )$, $g \in L^q \left ( \RR \right )$ oraz 
        $\frac{1}{p} + \frac{1}{q} = 1 + \frac{1}{r}$, $p,q,r > 1$. 
        
        Wykazać że:
        \[
            \norm{f * g}_r \leq \norm{f}_p \cdot \norm{g}_q 
        \]
    \end{zadanie*}

    Rozpoczniemy od szacowania na $\abs{(f * g)(x)}$:
    \begin{gather*}
        \abs{(f * g)(x)} \leq \int_{\RR} \abs{f(x - y) g(y)} dy =
        \int_{\RR} \abs{f(x-y)}^{1 + \frac{p}{r} - \frac{p}{r}} 
            \cdot \abs{g(y)}^{1+\frac{q}{r}-\frac{q}{r}} dy = \\
        \int_{\RR} \left ( \abs{f(x-y)}^p \cdot \abs{g(y)}^q \right )^{\frac{1}{r}}
            \cdot \abs{f(x-y)}^{\frac{r-p}{r}} \cdot \abs{g(y)}^{\frac{r-q}{r}} dy
    \end{gather*}
    zauważmy teraz że:
    \[
        \frac{1}{\frac{rp}{r-p}} + \frac{1}{\frac{rq}{r-q}} + \frac{1}{r} =
        \left ( \frac{1}{p} - \frac{1}{r} \right ) + 
            \left ( \frac{1}{q} - \frac{1}{r} \right ) + \frac{1}{r} =
        \frac{1}{p} + \frac{1}{q} - \frac{1}{r} = 1
    \]
    więc skorzystamy z uogólnionej nierówności Hölder'a dla tych właśnie wag:
    \begin{gather*}
        \leq \norm{\left ( f(x - y)^p  \cdot g(y)^q \right )^{\frac{1}{r}}}_r \cdot 
        \norm{\abs{f(x-y)}^{\frac{r - p}{r}}}_{\frac{rp}{r - p}} \cdot 
        \norm{\abs{f(y)}^{\frac{r - q}{r}}}_{\frac{rq}{r - q}} 
    \end{gather*}
    rozpiszmy każdy z tych czynników osobno:
    \begin{gather*}
        \norm{\left ( f(x - y)^p  \cdot g(y)^q \right )^{\frac{1}{r}}}_r 
        = \left ( \int_{\RR} \abs{f(x-y)}^p \abs{g(y)}^q dy \right )^{\frac{1}{r}}
        \\
        \norm{\abs{f(x-y)}^{\frac{r - p}{r}}}_{\frac{rp}{r - p}} =
        \left ( \int_{\RR} \abs{f(x-y)}^{\frac{r - p}{r} \cdot \frac{pr}{r-p}} \right )^
        {\frac{r-p}{pr}} =
        \norm{f}^{\frac{r - p}{r}}_p
        \\
        \norm{\abs{g(y)}^{\frac{r - q}{r}}}_{\frac{rq}{r - q}} =
        \left ( \int_{\RR} \abs{g(y)}^{\frac{r - q}{r} \cdot \frac{qr}{r-q}} \right )^
        {\frac{r-q}{qr}} =
        \norm{g}^{\frac{r - q}{r}}_q
    \end{gather*}
    zatem mamy że:
    \[
        \abs{(f * g)(x)} \leq \left ( 
            \norm{f}^{r-p}_p \cdot \norm{g}^{r-q}_q 
            \cdot \int_{\RR} \abs{f(x - y)}^p \abs{g(y)}^q  
        \right )^{\frac{1}{r}} 
    \]

    napiszmy w końcu że:
    \begin{gather*}
        \norm{f * g}^r_r = \left ( \int_{\RR} \abs{(f * g)(x)}^r  \right ) \leq
        \norm{f}^{r-p}_p \cdot \norm{g}^{r-q}_q \cdot
        \int_{\RR} \int_{\RR} \abs{f(x-y)}^p \abs{g(y)}^q dy \; dr 
    \end{gather*}

    skorzystajmy teraz z tw. Fubini'ego dla ostatniej całki:
    \begin{gather*}
        \leq \norm{f}^{r-p}_r \cdot \norm{g}^{r-q}_q \cdot 
        \int_{\RR} \abs{f(x)}^p dx \cdot   
        \int_{\RR} \abs{g(y)}^p dy \\
        = \norm{f}^{r-p}_p \cdot \norm{g}^{r-q}_q \cdot \norm{f}^{p}_p \cdot \norm{g}^q_q 
        = \norm{f}^r_p \cdot \norm{g}^r_q
    \end{gather*}
    co po spierwiastkowaniu daje tezę.
    
    \begin{zadanie*}
        Niech $\varphi \in C^{\infty}_c \left ( \RR \right )$. Obliczyć wartość granicy
        \[
            \lim_{n \to \infty } \lim_{\varepsilon \to 0^+} 
                \int_{\RR \backslash [- \varepsilon , \varepsilon]}
                \frac{n}{\sqrt{\pi } } \exp \left ( -n^2 x^2 \right ) \frac{\varphi (x)}{x} dl_1 (x)
        \]
    \end{zadanie*}

    Skorzystajmy z rozwinięcia Taylor'a dla $n = 1$ i mamy:
    \[
        \varphi (x) = \varphi (0) + x \varphi' (0) + \frac{x^2}{2!} f''(\xi)
        \lthen \frac{\varphi (x)}{x} = \frac{\varphi (0)}{x} + \varphi' (x) + \frac{x}{2} \varphi'' (\xi)
    \]

    zatem teraz nasza całka z zadania:
    \begin{gather*}
        \lim_{n \to \infty } \lim_{\varepsilon \to 0^+} 
            \int_{\RR \backslash [- \varepsilon , \varepsilon]}
            \frac{n}{\sqrt{\pi } } \exp \left ( -n^2 x^2 \right ) \frac{\varphi (x)}{x} dl_1 (x) = \\
        \lim_{n \to \infty } \lim_{\varepsilon \to 0^+} 
            \int_{\RR \backslash [- \varepsilon , \varepsilon]}
            \frac{n}{\sqrt{\pi } } \exp \left ( -n^2 x^2 \right ) 
            \cdot \left ( 
                \frac{\varphi (0)}{x} + 
                \varphi' (0) + 
                \frac{x}{2} \varphi''(\xi) 
            \right ) dl_1 (x) = \\
        \lim_{n \to \infty } \lim_{\varepsilon \to 0^+} 
            \int_{\RR \backslash [- \varepsilon , \varepsilon]}
            \frac{n}{\sqrt{\pi } } \exp \left ( -n^2 x^2 \right ) \frac{\varphi (0)}{x} dl_1 (x) + \\
        \lim_{n \to \infty } \lim_{\varepsilon \to 0^+} 
            \int_{\RR \backslash [- \varepsilon , \varepsilon]}
            \frac{n}{\sqrt{\pi } } \exp \left ( -n^2 x^2 \right ) \varphi' (0) dl_1 (x) + \\
        \lim_{n \to \infty } \lim_{\varepsilon \to 0^+} 
            \int_{\RR \backslash [- \varepsilon , \varepsilon]}
            \frac{n}{\sqrt{\pi } } \exp \left ( -n^2 x^2 \right ) \frac{x}{2} \varphi''(\xi) dl_1 (x)
    \end{gather*}
    rozważmy każdą z powyższych całek:
    \begin{enumerate}
        \item Pierwsza to całka z funkcji nieparzystej zatem jest równa 0. 
        \item Druga to całka z funkcja z funkcji $e^{-x^2}$ z odpowiednim skalowaniem 
            zatem jej wartość to $f'(0)$.
        \item Trzecią całkę możemy oszacować:
            \begin{gather*}
                \int_{\RR}
                    \frac{n}{\sqrt{\pi } } \exp \left ( -n^2 x^2 \right ) 
                    \frac{x}{2} \varphi''(\xi) dl_1 (x) \leq
                \int_{\RR}
                    \frac{n}{\sqrt{\pi } } \exp \left ( -n^2 x^2 \right ) 
                    \abs{\frac{x}{2} \varphi''(\xi)} dl_1 (x) \leq \\
                2M \cdot \int_0^{\infty}
                    \frac{n}{\sqrt{\pi } } \exp \left ( -n^2 x^2 \right ) \frac{x}{2} = 
                2M \cdot \frac{n}{\sqrt{\pi}} \cdot \frac{1}{2 n^2} \to 0
            \end{gather*}
    \end{enumerate}
    zatem w granicy mamy że:
    \begin{gather*}
        \lim_{n \to \infty } \lim_{\varepsilon \to 0^+} 
            \int_{\RR \backslash [- \varepsilon , \varepsilon]}
            \frac{n}{\sqrt{\pi } } \exp \left ( -n^2 x^2 \right ) \frac{\varphi (x)}{x} dl_1 (x) = \\
            0 + f'(0) + 0 = f'(0)
    \end{gather*}

    \newpage

    \begin{zadanie*}
        Wykazać, że:
        \begin{gather*}
            \mathcal{F} \left ( \frac{\sin x}{x} \right ) (k) = A \chi_{[- B, B]} (k)
        \end{gather*}
    \end{zadanie*}

    Zastosujmy twierdzenie o inwersji Fouriera:
    \[
        f(x) = \frac{1}{\sqrt{2 \pi } } \cdot \int_{\RR} A \chi_{[-B, B]} \cdot e^{i x k} dk = 
        \frac{A}{\sqrt{2 \pi } } \cdot \int_{[-B, B]} \css{x k} + i \snn{x k} dk =
    \]
    skorzystamy z faktu że $\sin$ jest funkcją nieparzystą a $\cos$ jest parzystą więc:
    \[
        = A \sqrt{\frac{2}{\pi }}  \cdot \int_0^B \css{x k} dk
    \]
    zastosujmy teraz podstawienie $a = x k$:
    \[
        = A \sqrt{\frac{2}{\pi }}  \cdot \frac{1}{x} \cdot \int_0^{x B} \cos a \; da 
        = A \sqrt{\frac{2}{\pi }}  \cdot \frac{\snn{x B}}{x}
    \]
    zatem porównując mamy że współczynniki $A, B$ to odpowiednio:
    \[
        \begin{cases}
            A = \sqrt{\frac{\pi }{2}} \\
            B = 1
        \end{cases}
    \]

    \newpage

    \begin{zadanie*}
        Definujemy:
        \[
            S := \left \{ f \in C^{\infty } (\RR) : 
                \forall_{k,l} \sup_{\RR} \abs{x^k f^{(l)} (x)} < + \infty   \right \}
        \]

        \begin{walk}
            \item Wykazać, że dla dowolnego $\varphi \in S$ istnieje granica
                \[
                    \lim_{\varepsilon \to 0^+} \int_{\RR \backslash [-\varepsilon, \varepsilon]} 
                        \frac{\varphi (x)}{x} dl_1 (x)  
                \]
            \item Wykazać, że:
                \[
                     \lim_{\varepsilon \to 0^+} \int_{\RR \backslash [-\varepsilon, \varepsilon]} 
                         \frac{1}{x} \cdot \left ( \mathcal{F} \varphi  \right ) (x) \; dl_1 (x) =
                         C \cdot \int_{\RR} \sgn (x) \cdot \varphi (x) \; dl_1(x)
                \]
                Znaleźć stałą C.
        \end{walk}

        \textit{Uwaga-wniosek:} "$F \left ( \frac{1}{x} \right ) = C \cdot \sgn (k)$" 
    \end{zadanie*}
    
    \begin{walk}
        \item Rozpiszmy:
            \begin{gather*}
                \lim_{\varepsilon \to 0^+} \int_{\RR \backslash [-\varepsilon, \varepsilon]} 
                    \frac{\varphi (x)}{x} dl_1 (x) = \\
                \lim_{\varepsilon \to 0^+} 
                    \int_{\RR \backslash [-1,1]} \frac{\varphi (x)}{x} dl_1 (x) +
                    \int_{[-1,1] \backslash [-\varepsilon, \varepsilon]} \frac{\varphi (x)}{x} dl_1 (x)
            \end{gather*}
            Obliczmy pierwszą całkę:
            \begin{gather*}
                \int_{\RR \backslash [-1,1]} \frac{\varphi (x)}{x} dl_1 (x) = 
                \int_{\RR \backslash [-1,1]} x \cdot \varphi (x) \cdot \frac{1}{x^2} dl_1(x) \leq \\
                \sup_{x \in \RR \backslash [-1, 1]} \left ( \abs{x \cdot \varphi (x)} \right ) \cdot  
                    \int_{\RR \backslash [-1,1]} \abs{\frac{1}{x^2}} dl_1 (x) \leq
                M \cdot (0 - (-1) + 1 - 0) = 2M
            \end{gather*}
            
            Przy drugiej całce skorzystamy z obserwacji:
            \[
                \int_{[-1,1] \backslash [-\varepsilon, \varepsilon]} \frac{\varphi (0)}{x} dl_1 (x) =
                \varphi (0) \cdot \int_{[-1,1] \backslash [-\varepsilon, \varepsilon]} 
                    \frac{1}{x} dl_1 (x) = 0 
            \]
            zatem:
            \begin{gather*}
                \int_{[-1,1] \backslash [-\varepsilon, \varepsilon]} \frac{\varphi (x)}{x} dl_1 (x) =
                \int_{[-1,1] \backslash [-\varepsilon, \varepsilon]} 
                    \frac{\varphi (x) - \varphi (0)}{x - 0} dl_1 (x)
            \end{gather*}
            oznaczmy przez $g(x) = \frac{\varphi (x) - \varphi (0)}{x - 0}$, i zauważmy że:
            \[
                g(x) = \frac{\varphi (x) - \varphi (0)}{x - 0} \to^{x \to 0} \varphi' (0)
            \]
            czyli definując $g(0) = \varphi' (0)$ mamy że $g(x) \in C (\RR)$ 
            (co oczywiście wynika z faktu że $\varphi (x) \in C^{\infty }(\RR)$)

            Teraz zostało powiedzieć że całka z funkcji ciągłej na przedziale zwartym 
            jest ograniczona, czyli niezależnie od $\varepsilon$ mamy:
            \begin{gather*}
                \int_{[-1,1] \backslash [-\varepsilon, \varepsilon]} \frac{\varphi (x)}{x} dl_1 (x) =
                \int_{[-1,1] \backslash [-\varepsilon, \varepsilon]} g(x) dl_1 (x) \\
                \leq \int_{[-1,1] \backslash [-\varepsilon, \varepsilon]} \abs{ g(x)} dl_1 (x) =
                \int_{[-1,1]} \abs{g(x)} dl_1 (x) = N
            \end{gather*}
            więc nasza granica:
            \begin{gather*}
               \lim_{\varepsilon \to 0^+} \int_{\RR \backslash [-\varepsilon, \varepsilon]} 
                   \frac{\varphi (x)}{x} dl_1 (x) \leq
                2M + N
            \end{gather*}
            
        \item Ponownie rozpiszmy:
            \begin{gather*}
                \lim_{\varepsilon \to 0^+} \int_{\RR \backslash [-\varepsilon, \varepsilon]} 
                    \frac{1}{x} \cdot \left ( \mathcal{F} \varphi  \right ) (x) \; dl_1 (x) =
                \lim_{\varepsilon \to 0^+} \int_{\varepsilon}^{\infty}
                    \frac{ 
                        \left ( \mathcal{F} \varphi  \right ) (x) - 
                        \left ( \mathcal{F} \varphi \right ) (-x) 
                    }{x} \; dl_1 (x)
            \end{gather*}

            zauważmy że:
            \begin{gather*}
                \left ( \mathcal{F} \varphi  \right ) (x) - \left ( \mathcal{F} \varphi \right ) (-x) =
                \frac{1}{\sqrt{2 \pi } } \int_{\RR} \varphi (k) e^{-ikx} \, dl_1 (k) - 
                \frac{1}{\sqrt{2 \pi } } \int_{\RR} \varphi (-k) e^{-ikx} \, dl_1 (k) = 
                \\
                \frac{1}{\sqrt{2 \pi } } \int_{\RR} \varphi (k) \cos (-kx) \, dl_1 (k) + 
                \frac{i}{\sqrt{2 \pi } } \int_{\RR} \varphi (k) \sin (-kx) \, dl_1 (k) \, \\ - 
                \frac{1}{\sqrt{2 \pi } } \int_{\RR} \varphi (-k) \cos (kx) \, dl_1 (k) -
                \frac{i}{\sqrt{2 \pi } } \int_{\RR} \varphi (-k) \sin (kx) \, dl_1 (k) \\ = 
                \frac{1}{\sqrt{2 \pi } } \int_{\RR} \varphi (k) \cos (kx) \, dl_1 (k) + 
                \frac{i}{\sqrt{2 \pi } } \int_{\RR} \varphi (k) \sin (-kx) \, dl_1 (k) \, \\ - 
                \frac{1}{\sqrt{2 \pi } } \int_{\RR} \varphi (k) \cos (kx) \, dl_1 (k) +
                \frac{i}{\sqrt{2 \pi } } \int_{\RR} \varphi (k) \sin (-kx) \, dl_1 (k) \\ = 
                \frac{2i}{\sqrt{2 \pi } } \int_{\RR} \varphi (k) \sin (-kx) \, dl_1 (k)
            \end{gather*}
            skorzystajmy teraz z tw. Fubiniego:
            \begin{gather*}
                \lim_{\varepsilon \to 0^+} \int_{\varepsilon}^{\infty}
                    \frac{ 
                        \left ( \mathcal{F} \varphi  \right ) (x) - 
                        \left ( \mathcal{F} \varphi \right ) (-x) 
                    }{x} \; dl_1 (x) = \\
                \lim_{\varepsilon \to 0^+} \int_{\varepsilon}^{\infty} \frac{1}{x}
                    \frac{2i}{\sqrt{2 \pi } } \int_{\RR} \varphi (x) \sin (-kx) \, dl_1 (k) \, dl_1 (x) = \\
                    -i \sqrt{\frac{2}{\pi}} \cdot \lim_{\varepsilon \to 0} \int_{\RR} \varphi (k) \cdot 
                    \int_{\varepsilon}^{\infty } \frac{1}{x} \cdot \sin (kx) \, dl_1 (x) dl_1(k)
            \end{gather*}
            Obliczmy wewnętrzną całkę korzystając z faktu, że bezproblemowo możemy 
            zabrać granicę pod całkę (z twierdzenia o Lebesgue'a o zbieżności zmajoryzowanej 
            bo $\varphi$ należy do $S$):
            \begin{gather*}
                \lim_{\varepsilon \to 0} \int_{\varepsilon}^{\infty } \frac{\sin (kx)}{x} dx = 
                \int_0^{\infty } \frac{\sin (kx)}{x} dx = \sgn (k) \int_0^{\infty } \frac{\sin x}{x} dx = 
                \frac{\pi }{2} \cdot \sgn (k)
            \end{gather*}
            Kończąc zapisujemy zatem że:
            \begin{gather*}
                   -i \sqrt{\frac{2}{\pi}} \cdot \lim_{\varepsilon \to 0} \int_{\RR} \varphi (k) \cdot 
                   \int_{\varepsilon}^{\infty } \frac{1}{x} \cdot \sin (kx) \, dl_1 (x) \; dl_1(k) = \\
                   -i \sqrt{\frac{2}{\pi}} \cdot \frac{\pi }{2} \cdot 
                       \int_{\RR} \varphi (k) \cdot \sgn (k) \, dl_1 (k) = 
                   -i \sqrt{\frac{\pi }{2}} \cdot \int_{\RR} \varphi (k) \cdot \sgn (k) \, dl_1 (k)
            \end{gather*}
        \end{walk}
 \end{document}
