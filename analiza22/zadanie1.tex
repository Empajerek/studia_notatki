\documentclass[11pt]{scrartcl}
\usepackage[sexy,noasy]{evan}
\author{Konrad Kaczmarczyk}
\usepackage{amsmath,systeme}
\usepackage{listings}
\usepackage[T1]{fontenc}
\usepackage{pgfplots}
\usepgfplotslibrary{colormaps}
\pgfplotsset{
    compat=1.18,
}
\begin{document}
    \title{Analiza II.2*}
    \subtitle{Rozwiązanie zadań z serii I}
    \maketitle
    \begin{zadanie*}
        Obliczyć wartość całki
        \[
            \int_{I_n} \max \left \{ x_1, \dots, x_n \right \} dl_n \left ( x_1, \dots , x_n \right )
        \]
        gdzie $I = [0,1]$
    \end{zadanie*}

    Podejdźmy do problemu probabilistycznie, niech $x_1, \dots x_n$ będą zmiennymi losowymi wybieranymi z zakeresu $[0,1]$.

    Łatwo widać że dla ustalonego $p \in [0,1]$ zachodzi:

    \[
        \mathbb{P} \left ( \sqrt[n]{x_1} < p   \right ) = p^n
    \]

    a także (wystarczy prosty argument polu wewnątrz $[0,p]^n$) że:

    \[
        \mathbb{P} \left ( \max \left \{ x_1, \dots, x_n  \right \} \leq p \right ) = p^n
    \]

    czyli funkcje $\max \left \{ x_1, \dots, x_n \right \}$, i $\sqrt[n]{x_1} $ mają te same dystrybuanty, a więc także wartości oczekiwane:
     \begin{gather*}
        \int_{I_n} \max \left \{ x_1, \dots, x_n \right \} dl_n \left ( x_1, \dots , x_n \right ) \\
        = \mathbb{E} \left [ \max \left \{ x_1, \dots, x_n \right \} \right ]
        = \mathbb{E} \left [ \sqrt[n]{x_1}  \right ] \\
        = \int_{I} \sqrt[n]{x_1} \; dl_1 (x_1) 
        = \left ( \frac{n}{n + 1} \right ) \cdot \left ( 1^{\frac{n}{n + 1}} - 0 \right ) = \frac{n}{n + 1}  
     \end{gather*}

     \newpage

     \begin{zadanie*}
        Obliczyć objętość obszaru ograniczonego powierzchnią
        \[
            S = \left \{ (x, y, z) : \left ( x^2 + y+2 + z^2 \right )^2 = a \left ( x^2 + y^2 - z^2 \right ) \right \}, \quad a > 0
        \]
     \end{zadanie*}

     W rozwiązaniu skorzystamy z zdania 10 z serii 1 na ćwiczenia (które również jest znane jako twierdzenie Pappus'a-Guldinus'a), zatem:

     Podstawiając współrzędne sferyczne mamy że $S$ to:
     \[
         S = \left \{ r \cdot \cos(\alpha ) \cos(\beta ), r \cdot \cos(\alpha ) \sin(\beta ) , r \sin(\alpha) \mid r^2 = a \cos (2 \alpha ) \right \}
     \]

     czyli region ograniczony $S$ jest bryłą obrotową. Podstawmy zatem $y = 0$ i oznaczmy:
     \[
         D = \left \{ \left ( x, y, z \right ) : \left ( x^2 + z^2 \right ) = a \left ( x^2 - z^2 \right ) \right \}
     \]

    Oznaczmy przez $P$ pole ograniczone $D$ i gdzie $x > 0$. Obliczmy wartość całki z $x$ w tym obszarze:

    \begin{gather*}
        \int_P x \; dl_2 = \int_{-\frac{\pi}{4}}^{\frac{\pi}{4}} \int_0^{\sqrt{a \cos 2\alpha } } r^2 \cos \alpha  \; dr \; d \alpha \\
        = \frac{2a^{\frac{3}{2}}}{3} \int_0^{\frac{\pi}{4}} \sqrt{\left ( \cos 2 \alpha  \right )^3} \cos \alpha  \; d \alpha 
        = \frac{2a^{\frac{3}{2}}}{3} \int_0^{\frac{\pi}{4}} \sqrt{\left ( 1 - 2 \sin^2 \alpha  \right )^3} \cos \alpha  \; d \alpha \\  
    \end{gather*}

    Dokonajmy podstawienia $s = \frac{1}{\sqrt{2} } \cdot \sin \alpha $:
    
    \begin{gather*}
         = \frac{2a^{\frac{3}{2}}}{3 \sqrt{2} } \int_0^1 \sqrt{\left ( 1 - s^2  \right )^3} \; ds \\  
    \end{gather*}

    Oraz ponownego podstawienia $s = \sin \delta$

    \begin{gather*}
        = \frac{2a^{\frac{3}{2}}}{3 \sqrt{2} } \int_0^{\frac{\pi}{2}} \sqrt{\left ( 1 - \sin^2 \delta  \right )^3} \cdot \cos \delta \; d \delta
        = \frac{2a^{\frac{3}{2}}}{3 \sqrt{2} } \int_0^{\frac{\pi}{2}} \cos^4 \delta \; d \delta \\  
        = \frac{2a^{\frac{3}{2}}}{3 \sqrt{2} } \cdot \frac{3}{16} = \frac{\pi \sqrt{2} }{16} a^{\frac{3}{2}} 
    \end{gather*}

    Zatem korzystając z twierdzenia Pappus'a - Guldinusa'a mamy że pole ograniczone krzywą daną w zadaniu jest:
    \[
        V = 2 \pi \cdot  \frac{\pi \sqrt{2} }{16} a^{\frac{3}{2}} = \frac{\pi^2 \sqrt{2} }{8} a^{\frac{3}{2}}
    \]

    \newpage

    \begin{zadanie*}
        Niech $A = (a_{ij})$ będzie macierzą rzeczywistą taką, że $\det A \not = 0$. Obliczyć n-wymiarową miarę Lebesgue'a zbioru
        \[
            B = \left\{ (x_1, \dots, x_n) : \sum_{i=1}^{n} \big| \sum_{j=1}^{n} a_{ij} x_j \big| \leq 1 \right\}
        \]
    \end{zadanie*}

    Rozpcznijmy od interpretacji wzoru w zadaniu. Można zauważyć że jest to złożenie przekształcenia zdefiniowanego macierzą $A$, z metryką taksówkową:
    \[
        \left\{ (x_1, \dots, x_n) : \sum_{i=1}^{n} \big| \sum_{j=1}^{n} a_{ij} x_j \big| \leq 1 \right\} 
        = \left \{ x : \norm{Ax} \leq 1 \right \} \quad 
        \mathrm{gdzie} \; x = (x_1, \dots, x_n)
    \]

    wprowadźmy oznaczenie:
    \[
        B' = \left \{ x : \norm{x} \leq 1 \right \}
    \]

    Pokażmy że $A^{-1} \left ( B' \right ) = B$ lub ekwiwalentnie $A(B) = B'$:
    \begin{enumerate}
        \item $x \in A(B)$ \\ czyli $\exists_y x = Ay$ gdzie $\abs{Ay} \leq 1$, więc mamy $\abs{x} \leq 1$ czyli $x \in B'$
        \item $x \in B'$ \\ czyli $\exists_y! x = Ay$(z odwracalności $A$), \\
            więc mamy $\abs{Ay} \leq 1$ i $y \in B$ czyli $x \in A(B)$
    \end{enumerate}


    możemy zastosować twierdzenie o zamianie zmiennych:
    \[
        l_n (B) = \int_B \, dl_n = \abs{\det A^{-1}} \int_{B'} \, dl_n = \abs{\det A^{-1}} \cdot l_n (B')  
    \]

    Wystarczy zatem policzyć objętość n-wymiarowej kuli w metryce taskówkowej, oznaczmy ją przez $B_n$. Narzuca się wzór indukcyjny:
    \[
        B_{n+1} = 2 \cdot \int_0^1 (1-x)^n B_n \; d \mu
        = 2 \cdot B_n \frac{1}{n+1} x^{n+1} \Big|_0^1
        = B_n \cdot \frac{2}{n + 1}
    \]

    Łatwo tutaj zauważyć jawny wzór:
    \[
        B_n = \frac{2^n}{n!}
    \]
    który spełnia rekurencje, a zatem wynikiem jest:
    \[
        l_n (B) = \frac{2^n}{n!} \cdot \frac{1}{\abs{\det A} }
    \]
    
    \newpage

    \begin{zadanie*}
        Niech $K \subset \RR^3$ będzie zbiorem wypukłym, zwartym, $f \in C^\infty \left ( \RR^3 \right )$ 
        spełnia warunek $\big|_{\partial K} \equiv 0$
        oraz $\left ( 0, 0, 0 \right ) \not \in \partial K$. Obliczyć wartość całki:
        \[
            \int_K \frac{1}{(x^2 + y^2 + z^2)^{3/2}} 
            \left( x \frac{\partial f}{\partial x} 
            + y \frac{\partial f}{\partial y} 
            + z \frac{\partial f}{\partial z} \right) 
            \, \mathrm{d} l_3 (x,y,z)
        \]
    \end{zadanie*}

    Zauważmy że:
    \begin{gather*}
        r \cdot \frac{\partial f}{\partial r} =
        r \cdot \frac{\partial x}{\partial r}  \frac{\partial f}{\partial x} +
        r \cdot \frac{\partial y}{\partial r}  \frac{\partial f}{\partial y} +  
        r \cdot \frac{\partial z}{\partial r}  \frac{\partial f}{\partial z} =
        x \cdot \frac{\partial f}{\partial x} +
        y \cdot \frac{\partial f}{\partial y} +  
        z \cdot \frac{\partial f}{\partial z}
    \end{gather*}
   
    Robimy zatem podstawienie sferyczne:
    \begin{gather*}
            \int_K \frac{1}{(x^2 + y^2 + z^2)^{3/2}} 
            \left( x \frac{\partial f}{\partial x} 
            + y \frac{\partial f}{\partial y} 
            + z \frac{\partial f}{\partial z} \right) 
            \, \mathrm{d} l_3 (x,y,z) \\
            = \int_{K'} \sin \alpha \frac{1}{r^3} r \cdot \frac{\partial f}{\partial r} \cdot r^2
            \, \mathrm{d} l_3 (r,\alpha , \beta ) 
            = \int_{K'} \sin \alpha  \frac{\partial f}{\partial r} \, \mathrm{d} l_3 (r,\alpha , \beta )
    \end{gather*}

    Mamy teraz dwa przypadki:
    \begin{walk}
        \item Punkt $(0, 0, 0)$ należy do $K$
        \item Punkt $(0, 0, 0)$ nie należy do $K$
    \end{walk}

    Pierwszy przypadek:
    Jeśli punkt $(0, 0, 0)$ należy do $K$, to całka jest równa:
    \begin{gather*}
        \int_{K'} \frac{\partial f}{\partial r} \sin \alpha \, \mathrm{d} l_2 (r,\alpha ) 
        = 2 \pi \cdot \int_0^{2\pi} \int_0^{r(\alpha)} \frac{\partial f}{\partial r} \sin \alpha  \, \mathrm{d} l_2 (r,\alpha ) \\ 
        = 2 \pi \cdot \int_0^{2\pi} \sin \alpha \int_0^{r(\alpha)} \frac{\partial f}{\partial r} \, \mathrm{d} l_2 (r,\alpha ) 
        = 2 \pi \cdot \int_0^{2\pi} \sin \alpha \left ( 0 - f(0) \right ) \mathrm{d} l_1 (\alpha) 
        = - 4 \pi \cdot f(0)
    \end{gather*}
    czyli wynik to $- 4 \pi f(0)$.

    Drugi przypadek:
    Jeśli punkt $(0, 0, 0)$ nie należy do $K$, to całka jest równa:
    \begin{gather*}
        \int_{K'} \frac{\partial f}{\partial r} \sin \alpha \, \mathrm{d} l_3 (r,\alpha , \beta ) 
        = 2 \pi \cdot \int_0^{2\pi} \int_{r_2(\alpha)}^{r_1(\alpha)} \frac{\partial f}{\partial r} \sin \alpha  \, \mathrm{d} l_2 (r,\alpha) \\
        = 2 \pi \cdot \int_0^{2\pi} \sin \alpha \left ( 0 - 0) \right ) \mathrm{d} l_1 (\alpha ) 
        = 0
    \end{gather*}
    czyli wynik w tym przypadku to $0$.
    
    
\end{document}
