\documentclass[11pt]{scrartcl}
\usepackage[sexy]{evan}
\author{Konrad Kaczmarczyk}
\usepackage{amsmath,systeme}
\usepackage{listings}
\usepackage[T1]{fontenc}
\begin{document}
  \title{GAL2 lato 2024}
  \subtitle{rozwiązania zadań z serii V}
  \maketitle
    \section{Zadanie}
        \begin{zadanie}
            W przestrzeni afinicznej euklidesowej $\mathbb{R}^3$ ze standardowym iloczynem skalarnym dane jest przekształcenie $\varphi_{r,s} : \mathbb{R}^3 \to \mathbb{R}^3$ wzorem
            \begin{gather*}
                \varphi_{(r,s)}(x_1, x_2, x_3) = \\
                \left( \frac{r}{3} x_1 - \frac{2}{3} x_2 - \frac{1}{3} x_3 + 1, 
                    \frac{2}{3} x_1 + \frac{1}{3} x_2 + \frac{2}{3} x_3 + 1, 
                    \frac{s}{3} x_1 + \frac{2}{3} x_3 - \frac{2}{3} x_3 + 1 
                \right),
            \end{gather*}
            gdzie $r,s \in \mathbb{R}$.

\begin{itemize}
    \item[a)] Dla jakich wartości parametrów $r, s \in \mathbb{R}$ przekształcenie $\varphi_{(r,s)}$ jest izometrią?
    \item[b)] Dany jest równoległościan $R \subseteq \mathbb{R}^3$ o objętości 54. Znaleźć objętość obrazu $\varphi_{(1,-3)}(R)$ tego równoległościanu przy przekształceniu $\varphi_{(1,-3)}$.
    \item[c)] Dla jakich wartości parametru $s \in \mathbb{R}$ przekształcenie $\varphi_{1,s}$ zmienia orientację ale zachowuje objętość 3-wymiarowych równoległościanów? Odpowiedź uzasadnić.
\end{itemize}
        \end{zadanie}
        
        \begin{walk}
            \item 
              \[
                \varphi_{(r,s)} (x_1, x_2, x_3 ) = \frac{1}{3} 
                \begin{bmatrix}
                  r & 2 & -1 \\
                  2  & 1 & 2 \\
                  s & 2 & -2
                \end{bmatrix}
                + 
                \begin{bmatrix}
                  0 & 1 & 1 \\
                \end{bmatrix}
              \]
              Wystarczy sprawdzić czy $\varphi'_{(r,s)}$ jest izometrią liniową, czyli czy
              \[
                 \begin{bmatrix}
                   r & 2 & -1 \\
                   2  & 1 & 2 \\
                   s & 2 & -2
                 \end{bmatrix}^T
                 \cdot 
                 \begin{bmatrix}
                   r & -2 & -1 \\
                   2 & 1 & 2  \\
                   s & 2 & -2
                 \end{bmatrix}
                 =
                 \begin{bmatrix}
                   9 & 0 & 0 \\
                   0 & 9 & 0  \\
                   0 & 0 & 9
                 \end{bmatrix}
              \]
              i po przeliczeniu mamy, że zachodzi to tylko dla 
              \[
                  \begin{cases}
                      r = 2 \\
                      s = 1
                  \end{cases}
              \]
              i widzimy że wtedy $\varphi'$ jest izomorfizmem, czyli $\varphi$ też jest, więc z definicji tylko  $\varphi_{(2,1)}$ jest izometrią afiniczną.
            \item Korzystając z faktów udowodnionych na wykładzie wystarczy obliczyć
              \[
                54 \cdot \abs{\text{det}\left ( \varphi'_{(1,-3)} \right )} = 54 \cdot \frac{1}{3} = 18 
              \]
            \item Podobnie jak porzednio wnioskujemy że $\abs{\text{det}\left ( \varphi_{(1,s)} \right )} = 1 $ oraz wiemy że przekształcenie ma zmieniać orientacje (np. bazy standardowej), czyli $\text{det}\left ( \varphi_{(1,s)} \right ) < 0$, mamy więc że 
              \[
                \text{det} \left ( \varphi'_{(1,s)} \right ) = -1 \Rightarrow -\frac{1}{9} \left ( s+6 \right ) = -1 \Rightarrow s + 6 = 9 \Rightarrow s = 3
              \]
              
        \end{walk}
        
        \section{Zadanie}
            \begin{zadanie}
              Rozważamy $\mathbb{R}^2$ ze standardowym iloczynem skalarnym. Niech $A = \begin{pmatrix} 1 & 3 \\ 2 & 6 \end{pmatrix}$.

\begin{itemize}
    \item[(a)] Podać wzory na dwa różne $f, g \in \text{End}(\mathbb{R}^2)$ samosprzężone oraz bazy $A, \mathcal{B}$ takie, że $A = M(f)_A^A = M(g)_\mathcal{B}^\mathcal{B}$.
    \item[(b)] Opisać wszystkie wektory $v \in \mathbb{R}^2$ takie, że $\mathcal{C} = \{(1,0), v\}$ jest bazą $\mathbb{R}^2$ i $f \in \text{End}(\mathbb{R}^2)$ dany warunkiem $M(f)_\mathcal{C} = A$ jest samosprzężone.
\end{itemize}
            \end{zadanie}
            
            \begin{walk}
                \item Wystarczy znaleźć macierze podobne do $A$ które dodatkowo są symetryczne (aby przekształcenia które definiują były samosprzężone) dla przykładu mogą to być
                  \[
                     \begin{bmatrix}
                         0 & 0 \\
                         0 & 7 \\
                     \end{bmatrix}
                     \qquad \qquad 
                     \begin{bmatrix}
                         1 & \sqrt{6}  \\
                         \sqrt{6}  & 6 \\
                     \end{bmatrix}
                  \]
                  oraz bazy im odpowiadające to 
                  \[
                      \mathcal{A} = \left \{ \left ( - \frac{2}{7} , \frac{1}{7} \right ) , \left ( \frac{1}{7} , \frac{3}{7} \right ) \right \}
                      \qquad 
                      \mathcal{B} = \left \{ \left ( \frac{13}{6} \sqrt{6}, -1  \right ) , \left ( - \frac{1}{2} \sqrt{6} , 4  \right ) \right \} 
                  \]
                \item Niech $v = (a,b)$ żeby razem z $\left ( 1 , 0 \right )$ było bazą wystarczy że $b \not = 0$, a żeby endomofizm był samosprzężony wystarczy że 
                  \[
                      \begin{bmatrix}
                         1  & a \\
                         0  & b \\
                      \end{bmatrix}
                      \cdot 
                      \begin{bmatrix}
                          1 & 3 \\
                          2 & 6 \\
                      \end{bmatrix}
                      \cdot 
                      \begin{bmatrix}
                          1 & -\frac{a}{b} \\
                          0 & \frac{1}{b} \\
                      \end{bmatrix}
                      =
                      \frac{1}{b}
                      \begin{bmatrix}
                          \cdots  & 3 + 5a - 2a^2 \\
                          2 b^2 & \cdots  \\
                      \end{bmatrix}
                  \]
                  było symetryczne, czyli $2b^2 = 3 + 5a - 2a^2$. Jest to równanie okręgu,
                  \[
                      \left ( a - \frac{5}{4} \right )^2 + b^2 = \left ( \frac{7}{4} \right )^2
                  \]
                  i dowolny punkt (poza $a = 3 , b = 0 $, i $a = - \frac{1}{2}, b = 0 $), spełnia warunki.
            \end{walk}
            \newpage
      \section{Zadanie}
          \begin{zadanie}
              Forma dwuliniowa $h : \mathbb{R}^4 \times \mathbb{R}^4 \to \mathbb{R}$ dana jest warunkiem
\[
G(h, st) = \begin{pmatrix}
0 & -1 & 0 & 1 \\
-1 & -1 & 0 & 1 \\
0 & 0 & 0 & -1 \\
1 & 1 & -1 & -1
\end{pmatrix},
\]
gdzie $st = \{\varepsilon_1, \varepsilon_2, \varepsilon_3, \varepsilon_4\}$ jest bazą standardową przestrzeni $\mathbb{R}^4$.

\begin{itemize}
    \item[(a)] Znaleźć bazę prostopadłą przestrzeni dwuliniowej $(\mathbb{R}^4, h)$ oraz znaleźć sygnaturę macierzy $G(h, st)$.
    \item[(b)] Czy w przestrzeni $W = \text{lin}(\varepsilon_2, \varepsilon_4) \subseteq \mathbb{R}^4$ jest niezerowy wektor izotropowy formy $h$? Czy istnieje dwuwymiarowa podprzestrzeń $Z \subseteq \mathbb{R}^4$ złożona z wektorów izotropowych formy $h$? Odpowiedzi uzasadnić.
\end{itemize}
          \end{zadanie}

          \begin{walk}
              \item Znajdowanie bazy prostopadłej jest procesem algorytmicznym,
                \begin{itemize}
                    \item Wybieramy wektor prostopadły do już wybranych
                    \item Sprawdzamy czy przypadkiem nie jest izotropowy 
                    \item Jesli nie, to dodajemy ten wektor do bazy
                    \item Powtarzamy proces
                \end{itemize}
                
                Zatem zaczynamy:
                Niech $v_1 = (0,0,0,1)$, sprawdzamy czy nie jest izotropowy (nie jest), i sprawdzamy warunek na prostopadłosć z $v_1$: $x_{1} + x_2 - x_3 - x_4 = 0$, wybieramy zatem np. $v_2 = (1, -1, 0, 0)$ (nie jest izotropowy), mamy równanie $x_1 = 0$, i powtarzając do końca mamy bazę $\mathcal{A} = \left \{ \left ( 0,0,0,1 \right ), \left ( 1,-1,0,0 \right ), \left ( 0,0,1,-1 \right ), \left ( 0,1,1,0 \right ) \right \} $, i wtedy nasza forma to 
                \[
                    G(h, \mathcal{A} ) = 
                    \begin{pmatrix}
                      -1 & 0 & 0 & 0 \\
                      0  & 1 & 0 & 0 \\
                      0 &  0 & 1 & 0 \\
                      0 & 0 & 0 & -1
                    \end{pmatrix}
                \]
                więc sygnatura $A$ to 
                \[
                    s(A) = 0
                \]
            \item Dla $W$ mamy formę:
              \[
                G(h\big|_{W}, W) = 
                \begin{bmatrix}
                    -1 & 1 \\
                    1 & -1 \\
                \end{bmatrix}
              \]
              więc dla wektora $v = (a,b)$, aby był izotropowy wystarczy
              \[
                  \begin{bmatrix}
                      a & b \\
                  \end{bmatrix}
                  \cdot 
                  \begin{bmatrix}
                      -1 & 1 \\
                      1 & -1 \\
                  \end{bmatrix}
                  \cdot 
                  \begin{bmatrix}
                      a \\
                      b \\
                  \end{bmatrix}
                  = 
                  -a^2 + 2ab -b^2 = 0
              \]
              więc $a = b$, czyli $v := (1,1)$ jest izotropowy. 

              Wystarczy więc znaleźć wektor $w \bot v$, który jednoczesnie jest izotropowy.
              Aby spełnić pierwszy warunek wystarczy że $x_3 = 0$, a patrząc na przekątną macierzy $G(h,st)$, 
              narzuca się kandydat $w = \epsilon_1$, który spełnia warunki. 
              Zatem $Z = \text{lin} \left ( \left ( 1,0,0,0 \right ), (0,1,0,1) \right )$, 
              w której wszystkie wektory są izotropowe, macierz formy $h$ w niej jest zerowa.
          \end{walk}
          
\end{document}
