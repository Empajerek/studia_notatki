\documentclass[11pt]{scrartcl}
\usepackage[sexy]{evan}
\author{Konrad Kaczmarczyk}
\usepackage{amsmath,systeme}
\usepackage{listings}
\usepackage[T1]{fontenc}
\begin{document}
  \title{GAL2, lato 2024}
  \subtitle{notatki z ćwiczeń}
  \maketitle
  \section{cwiczenia 26 luty}
      \begin{definicja}
          $A \in M_n (K)$ jest skalarna, jesli $A = aI$, $a \in K$
      \end{definicja}

      \begin{fakt}
          $A$ jest skalarna $\iff $ jedyna macierz w $M_n(k)$ podobna do A jest A.
      \end{fakt}
      
      \begin{enumerate}
          \item $\Rightarrow $

          \item $\Leftarrow $
            Załóżmy że $A$ nie jest taka, że $a_{ij} \not = 0$ dla jakis $i, j$.
            \begin{uwaga}
              \[
                I_i(c) = \begin{bmatrix}
                           1 & 0 & \cdots \\
                           \cdots & c & \cdots \\
                           0 & \cdots & 1
                        \end{bmatrix}
              \]
            \end{uwaga}
            Więc $A' = I_i(c) \cdot A \cdot I_i(\frac{1}{c}) \not A$
      \end{enumerate}
  \section{Endomorfizmy}
      \begin{zadanie}
          Niech $W \subset V$, $dim W = n$, $dim V = 2n$, $n \geq 1$ \\
          \[
              \exists f \in End(V) \qquad (f: V \to V)
          \]
          taki, że: $ker (f) = im (f) = W$
           
      \end{zadanie}

      Wystarczy zdefiniować na bazie.
       
      \begin{zadanie}
          $f,g \in End(V) \quad 
          \begin{cases}
              r(f) = r(g) = 1 \\
              Ker(f) = Ker(g) \\
              Im(f) = Im(g)
          \end{cases}
          \Rightarrow f \circ g = g \circ f $
      \end{zadanie}

      \begin{zadanie}
          Niech $A$ będzie macieżą nieodwracalną. Pokaż że jest podobna do jakiejs macierzy z zerowym wierszem.
      \end{zadanie}
      
      Skoro $ det A = 0 $ to $ dim ker \phi \geq 1 $, a stąd $ \exists v \not = 0 $ taki że $ \phi (v) = 0$ z tw. Steiniza o wymianie możemy uzupełnic zbiór $\{ v \}$ do bazy nazwijmy $A$.

      Drugie rozwiązanie:
      Z faktu że rzędy są linowo zależne, z pomocą operacji elementarnych możemy wyzerować jeden z wierszy tworząc macierz z zerowym wierszem.

  \section{ćwiczenia 4 marca}
    Rozwiązania zadań:

    \begin{zadanie}
        Dla każdego z endomorfizmów $\varphi$ znaleźć wartosci własne i bazy ich przestrzeni własnych
        \begin{walk}
            \item $\phi : \mathbb{R}^2 \to \mathbb{R}^2, \varphi \left ( x_{1}, x_{2} \right ) = \left ( 2 x_{1} - x_{2}, -x_{1} + 2x_{2} \right )$.
            \item $\varphi : \mathbb{R}^2 \to \mathbb{R}^2, \varphi \left ( x_{1}, x_{2} \right ) = \left ( 5x_{1} - x_{2}, -x_{1} + 2x_{2} \right )$.
            \item $\varphi : \mathbb{R}^3 \to \mathbb{R}^3 , \varphi \left ( x_{1}, x_{2}, x_{3} \right ) = \left ( 4x_{1} + x_{2}, 3x_{1} + 2x_{2}, 7x_{1} - 7x_{2} + 5x_{3} \right )$.
            \item $\varphi : \mathbb{R}^3 \to \mathbb{R}^3, \varphi \left ( x_{1}, x_{2}, x_{3} \right ) = \left ( x_{1} - x_{2}, x_{1} + 3x_{2} + x_{3}, 2x_{3} \right )$ 
            \item $\varphi : \mathbb{R}^4 \to \mathbb{R}^4, \varphi \left ( x_{1}, x_{2}, x_{3}, x_{4} \right )= $ \\
              $\left ( -6x_{1} - x_{2} + 2x_{3}, 3x_{1} + 2x_{2} + x_{4}, -14x_{1} - 2x_{2} + 5x_{3}, -x_{4} \right )$
        \end{walk}
    \end{zadanie}

    \begin{walk}
        \item Weźmy bazę $\mathcal{A}  = \left \{ \left ( 1, 1 \right ), \left ( 1, -1 \right ) \right \}$, dla której macierz przekształcenia $\varphi$ wygląda następująco, 
          \[
            M \left ( \varphi \right )^{\mathcal{A} }_{\mathcal{A} } = 
            \begin{bmatrix}
                1 & 0 \\
                0 & 3 \\
            \end{bmatrix}
          \]
          więc wartosci własne to 1, i 3, a bazą podprzestrzeni własnej jest $\mathcal{A} $.
        \item Wielomian charakterystyczny w bazie standardowej to 
          \[
              \left ( 5 - \lambda \right ) \cdot \left ( 2 - \lambda \right ) - \left ( -1 \right )\left ( 1 \right ) = 11 - 7 \lambda + \lambda^2
          \]
          co po rozwiązaniu równania kwadratowego daje nam dwie wartosci własne: $\frac{7 + \sqrt{5} }{2}$, oraz $\frac{7 - \sqrt{5} }{2}$, które po dalszych obliczeniach dają wektory własne $\left ( \frac{-3 + \sqrt{5} }{2}, 1 \right )$, i $\left ( \frac{-3 - \sqrt{5} }{2}, 1 \right )$, które przy okazji są bazą podprzesterzeni własnej.
        \item Łatwo zauważyć że $\varphi \left ( \left ( 0, 0, 1 \right ) \right ) = \left ( 0, 0, 5 \right )$, czyli znaleźlismy wektor własny i jego wartosc własną (przyjmijmy że znajduje się w bazie podprzestrzeni własnej), zauważyć można jeszcze że wektorem własnym jest $\left ( 1,1 \right )$, z wartoscią 5. Rozwiązując teraz wielomian charakterystyczny otrzymujemy ostatnią wartosć własną czyli $1$, więc znaleźć metodą macierzową pozostały wektor $\left ( 1, -3 \right )$.
        \item Postępując podobnie mamy: $\lambda = 2$, $v_{1} = \left ( 1, -1, 0 \right ) $
        \item Analogicznie mamy:
          \begin{gather*}
            \left ( v_{1}, v_{2}, v_{3} \right ) = \left ( \left ( 1, -3, 2, 0 \right ) \right )
          \end{gather*}
          ze swoimi wartosciami własnymi kolejno $-1$, i $1$.
    \end{walk}
    
   \begin{zadanie}
        Niech $V$ będzie przestrzeni bazą funkcji $\mathbb{R}  \to \mathbb{R} $ mających pochodne i-tego stopnia dla każdego $i \in \mathbb{N} $ i niech $\phi : V \to V$ będzie różniczkowaniem, to znaczy $\phi \left ( f \right ) = f'$ dla każdego $f \in V$. Wykazać, że każda liczba $a \in \mathbb{R} $ jest wartoscią własną endomorfizmu $\phi $. Dla każdego $a \in R$ znaleźć $V_{(a)}$.
   \end{zadanie}
    
   Rozwiązanie zadania sprowadza się do rozwiązania równiania:
   \[
       f' = \phi \left ( f \right ) = af
   \]
   czyli $f = c \cdot e^{ax}$.
   Dodatkowo można zauważyć że $V_{(a)} = \left ( \left ( e^{ax} \right ) \right )$.
   
   \begin{zadanie}
     \begin{walk}
     \item Niech $\mathcal{A} = \left \{ \alpha_{1}, \alpha_{2}, \alpha_{3}, \cdots, \alpha_{n} \right \} $ będzie bazą podprzestrzeni $V$, niech $ \varphi: V \to V$, z warunkiem $\varphi \left ( \alpha_{i} \right ) = \alpha_{i+1}$, oraz
       \[
         \varphi \left ( \alpha_n \right ) = a_{0} \alpha_{1} + \cdots + a_{n-1} \alpha_{n}
       \]
       Znaleźć wielomian charakterystyczny.
      
    \item Wykazać że dla wielomianu w postaci:
      \[
        w \left ( \lambda \right ) = \sum_{i = 0}^{n} a_i \lambda_i \qquad a_n = (-1)^n
      \]
      o współczynnikach w ciele $K$ istnieje macierz $A \in M_n \left ( K \right )$, taka że $w$ jest jej wielomianem charakterystycznym.
    \end{walk}
   \end{zadanie}
   
   \begin{walk}
       \item Wystarczy napisać macierz i ze wzorów Laplace'a mamy że:
         \[
           det \left ( A \right ) = (-1)^{n-1} a_{0} + \lambda \left ( -1 \right )^{n-2} + \cdots + \lambda_{n-1} a_{n-1} 
         \]
      \item Wystarczy rozpatrzeć macierz w postaci
        \[
            \begin{bmatrix}
              0 & 1 & 0 & \cdots  \\
              0 & 0 & 1 & \cdots  \\
              \vdots & \vdots  & \vdots  & \vdots \\
              -q_{1} & -q_{2} & -q_{3} & \cdots 
            \end{bmatrix}
        \]
   \end{walk}
   
   \begin{zadanie}
       Czy istnieje endomorfizm, który ma conajmniej jeden wektor własny, ale nie ma ich niekskończnie wiele?
   \end{zadanie}
   
   Wtedy i tylko wtedy gdy ciało jest skończone.

   \section{5 marca}
       

   \begin{zadanie}
       Dla każdego z endomorfizmów $ \varphi  : V \to V$ zbadać czy istnieje baza $\mathcal{A} $ przestrzeni $V$ złożona z wektorów własnych $\varphi$ (czytaj jest diagonalizowalna). Jesli tak to podaj przykład takiej bazy oraz $\text{M} \left ( \varphi \right )_{\mathcal{A}}^{\mathcal{A}}$.
   \end{zadanie}
   
   \begin{walk}
       \item Licząc wartosci własne macierzy endomorfimu $
         \begin{bmatrix}
             1 & -1 \\
             1 & 3 \\
         \end{bmatrix}
         $
         znajdujemy że ma ona jeden wektor własny czyli $\left ( 1, -1 \right )$, więc nie ma bazy.
       \item Pdobnie jak w zadaniu 3.1 a, wektorami własnymi są $\left ( 0,1,1 \right )$ i $\left ( 0, 1, -1 \right )$, i szybko licząc wielomian mamy pozostały wektor $\left ( 1,1, 0 \right )$, które razem rozpinają $\mathbb{R}^3$, więc są bazą $\mathcal{A} $,  macierz to:
         \[
           \text{M} \left ( \varphi \right )_{\mathcal{A}}^{\mathcal{A}} = 
             \begin{bmatrix}
               1 & 0 & 0 \\
              0 & 3 & 0 \\
               0 & 0 & 3 
             \end{bmatrix}
         \]
      \item Macierz $\text{M} \left ( \varphi \right )_{id}^{id}$, jest w postaci blokowej, więc możemy ją rozbić na dwie pomniejsze, zatem w tej pierwszej mamy wektory $\left ( 1, -1 \right )$, i $\left ( \frac{1}{5}, \frac{1}{4} \right )$,  w drugiej są $\left ( 1,1 \right )$, oraz $\left ( 1, -3 \right )$, które równierz rozpinają i razem z poprzednimi tworzą bazę.
   \end{walk}

   \begin{zadanie}
       Niech
       \[
           A_{1} = 
           \begin{bmatrix}
             1 & 2 & 0 \\
             2 & -2 & 0\\
             0 & 0 & -3
           \end{bmatrix}
           \quad 
           A_{2} = 
           \begin{bmatrix}
             -3 & 1 & 1 \\
             0 & 1 & 2 \\
             0 & 2 & -2
           \end{bmatrix}
           \quad 
           A_{3} = 
           \begin{bmatrix}
             0 & 2 & 0 \\
             -2 & 0 & 0 \\
             0 & 0 & 2
           \end{bmatrix}
           \quad 
           A_{4} = 
           \begin{bmatrix}
             0 & 1 & 0 \\
             -4 & 4 & 0 \\
             -2 & 1 &2
           \end{bmatrix}
       \]
       Dla każdej z powyższych macierzy $A_i$, $i = 1, \dots ,4$ zbadać czy $A_i$ jest diagonalizowalna nad $\mathbb{R} $ oraz czy $A_i$ jest diagonalizowalna nad $\CC$
   \end{zadanie}
   \begin{enumerate}
       \item Łatwo zauważyć wektory własne: $\left ( 0,0,1 \right )$, $\left ( 2,1,0 \right )$, $\left ( 1, -2 \right )$, które rozpinają $\mathbb{R}^3$, i należą do $\QQ^3$, zatem jest diagonalizowalna nad $\QQ$.
       \item Podobnie znajdujemy wektor $\left ( 1, 0, 0 \right )$, i licząc dalej znajdujemy kolejny wektor własny $\left ( 2, 10, 5 \right )$, lecz tylko jeden więc nie jest diagonalizowalna.
        \item \dots 
        \item \dots
   \end{enumerate}
   

   \begin{zadanie}
       Podaj przykład macierzy $A \in M_2 \left ( \QQ \right )$, która nie jest diagonalizowana nad $\QQ$ a jest nad $\mathbb{R}$ .
   \end{zadanie}

   Weżmy macierz Fibo czyli
   \[
       \begin{bmatrix}
           1 & 1 \\
           1 & 0 \\
           
       \end{bmatrix}
   \]
  Która ma wartosci własne niewymierne.

  \begin{zadanie}
      Wykaż że jesli macierz $A \in M_n \left ( K \right )$, ma dokładnie jedną wartosć własną, oraz jest diagonalizowalna to jest macierzą diagonalną.
  \end{zadanie}
  
  Po prostej kalkulacji:
  \[
    A = C \lambda \text{Id} C^{-1} = \lambda \text{Id}  
  \]

  \begin{zadanie}
      Niech $\varphi$ będzie odwracalnym endomorfizmem. Wykaż że jesli $\varphi$ jest diagonalizowalym endomorfizmem to ${\varphi}^{-1} $ też jest.
  \end{zadanie}
  
  Przez kalkulację:
  \[
    A = C D C^{-1} \Rightarrow  {A}^{-1} = (CDC^{-1})^{-1} = C D' C^{-1}    
  \]
  
  \begin{zadanie}
      Czy $f : M_n \left ( \mathbb{R}  \right ) \to M_n \left ( \mathbb{R}  \right )$, $f(A) = A^T$ jest diagonalizowalny? Jesli tak to podać bazę $\mathcal{A} $ w której macierz przekształcenia $f$ jest diagonalna i znaleźć $\text{M} \left ( f \right )_{\mathcal{A}}^{\mathcal{A}}$.
  \end{zadanie}
  
  Zadanie już się pojawiło jako 6 z drugiej serii.

  \begin{zadanie}
      Wykazać że $A \in M_n \left ( K \right )$, jest diagonalizowalna wtedy gdy $A^T$ też jest. Podaj przykład gdy nie mają tych samych wektorów własnych.
  \end{zadanie}
  
  Podobnie jak w zadaniu 4.5
  \[
    A = C D C^{-1} \Rightarrow  {A}^{T} = {CDC^{-1}}^{T} = {C^T}^{-1} D C^T     
  \]

  \section{11 marca}
      \begin{zadanie}
          \begin{walk}
              \item Niech $A \in M_n(K)$. Wykazac, ze dla kazdej wartosci własnej $\lambda_i$, $\dim V( \lambda_i )$ równa się liczbie klatek Jordana odpowiadających wartosci własnej $\lambda_i$ w postaci Jordana $A$.
              \item Niech $n \in \{1, 2, 3\}$ i niech $A \in M_n(K)$ będzie taka, że $w_A(\lambda) = (\lambda_{1} - \lambda)^n$. Wykazać, że $\dim V(\lambda_{1})$ wyznacza postać Jordana A.
              \item Czy 1b) jest prawdą dla $n \geq 4$?
              \item Niech $A \in M_n(K)$ będzie taka, że $w_A(\lambda) = (\lambda_{1} - \lambda)_{k1} \cdot  (\lambda_r - \lambda)_{kr}$ , gdzie $k_i \in {1, 2, 3}$ dla wszystkich $i = 1, \dots  , r$. Wykazać, że wszystkie $\dim V(\lambda_i )$ razem wyznaczają postać Jordana A.
          \end{walk}
          
      \end{zadanie}

      \begin{walk}
      \item Wystarczy zapisać $\dim V_{(a)} = n - r(A - aI) = \dim A - aI$.
      \item \dots 
      \end{walk}
    
      
      \begin{zadanie}
          Niech
          \[
              A_{1} = 
              \begin{bmatrix}
                2 & 0 & 1 \\
                1 & 2 & 0 \\
                0 & 0 & 2 \\
              \end{bmatrix}
              \qquad 
              A_{2} = 
              \begin{bmatrix}
                2 & 0 & 0 \\
                1 & 2 & 1 \\
                0 & 0 & 2
              \end{bmatrix}
              \qquad 
              A_{3} = 
              \begin{bmatrix}
                2 & 0 & 0 \\
                0 & 2 & 0 \\
                0 & 1 & 2
              \end{bmatrix}
              \qquad 
              A_4 = 
              \begin{bmatrix}
                2 & 0 & 0 \\
                1 & 2 & 0 \\
                1 & 1 & 2
              \end{bmatrix}
          \]
          Sprawdź które z tych macierzy są podobne. 
      \end{zadanie}

      Wystarczy znaleźć bazy Jordana i porównać, z czego wynika tylko że $A_1 \sim A_4$
      
\end{document}
